\documentclass[12pt]{amsart}
\usepackage{amstext,amsfonts,amssymb,amscd,amsbsy,amsmath,verbatim}
\usepackage{ifthen}
\usepackage{color,tikz}

\usepackage{amsthm}
\usepackage{latexsym}
\usepackage[all]{xy}
\usepackage{enumerate}
\usepackage{url}

\newtheorem{lemma}{Lemma}[section]
\newtheorem{theorem}[lemma]{Theorem}
\newtheorem{propo}[lemma]{Proposition}

\newtheorem{prop}[lemma]{Proposition}
\newtheorem{cor}[lemma]{Corollary}
\newtheorem{conj}[lemma]{Conjecture}
\newtheorem{claim}[lemma]{Claim}
\newtheorem{claim*}{Claim}
\newtheorem{thm}[lemma]{Theorem}
\newtheorem{defn}[lemma]{Definition}
\newtheorem{example}[lemma]{Example}
\newtheorem{definition}[lemma]{Definition}
\newtheorem{problem}[lemma]{Problem}
\newtheorem*{problem*}{Problem}


\theoremstyle{remark}
\newtheorem{remark}[lemma]{Remark}

\usepackage{geometry,enumerate}
\geometry{a4paper, top=3.5cm, bottom=3cm, left=3cm, right=3cm}

\parindent = 6pt
\parskip = 4pt

% Commands
\newcommand{\isom}{\cong}
\newcommand{\m}{\mathfrak m}
\newcommand{\lideal}{\langle}
\newcommand{\rideal}{\rangle}
\newcommand{\initial}{\operatorname{in}}
\newcommand{\Hilb}{\operatorname{Hilb}}
\newcommand{\Spec}{\operatorname{Spec}}
\newcommand{\im}{\operatorname{im}}
\newcommand{\NS}{\operatorname{NS}}
\newcommand{\Frac}{\operatorname{Frac}}
\newcommand{\ch}{\operatorname{char}}
\newcommand{\Proj}{\operatorname{Proj}}
\newcommand{\id}{\operatorname{id}}
\newcommand{\Div}{\operatorname{Div}}
\newcommand{\tr}{\operatorname{tr}}
\newcommand{\Tr}{\operatorname{Tr}}
\newcommand{\Supp}{\operatorname{Supp}}
\newcommand{\Gal}{\operatorname{Gal}}
\newcommand{\Pic}{\operatorname{Pic}}
\newcommand{\QQbar}{{\overline{\mathbb Q}}}
\newcommand{\Br}{\operatorname{Br}}
\newcommand{\Bl}{\operatorname{Bl}}
\newcommand{\cF}{\mathcal{F}}
\newcommand{\NN}{\mathbb{N}}
\newcommand{\grad}{\nabla}
\DeclareMathOperator*{\argmin}{arg\,min}

\newcommand{\Cox}{\operatorname{Cox}}
\newcommand{\Tor}{\operatorname{Tor}}
\newcommand{\diam}{\operatorname{diam}}
\newcommand{\Hom}{\operatorname{Hom}} %done
\newcommand{\sheafHom}{\mathcal{H}om}
\newcommand{\Gr}{\operatorname{Gr}}
\newcommand{\Osh}{{\mathcal O}}
\newcommand{\kk}{\kappa}
\newcommand{\rank}{\operatorname{rank}}
\newcommand{\codim}{\operatorname{codim}}
\newcommand{\conv}{\operatorname{conv}}
\newcommand{\D}{{\mathcal D}}
\newcommand{\PP}{\mathbb{P}}
\newcommand{\EE}{\mathbb{E}}

\newcommand{\RR}{\mathbb{R}}
\newcommand{\zz}{\mathbb{Z}}
\newcommand{\Sym}{\operatorname{Sym}} %done
\newcommand{\GL}{{GL}}
\newcommand{\grG}{{\mathcal{G}}}

\newcommand{\Syz}{\operatorname{Syz}}
\newcommand{\defi}[1]{\textsf{#1}} % for defined terms


\newcommand{\Bmod}{\ensuremath{B_
\text{mod}}}
\newcommand{\Bint}{\ensuremath{B_\text{int}}}
\newcommand\commentr[1]{{\color{red} \sf [#1]}}
\newcommand\commentb[1]{{\color{blue} \sf [#1]}}
\newcommand\commentm[1]{{\color{magenta} \sf [#1]}}
\newcommand{\ddr}[1]{{\color{blue} \sf $\clubsuit\clubsuit\clubsuit$ Daniel: [#1]}} 
\newcommand{\mv}[1]{{\color{red} \sf $\clubsuit\clubsuit\clubsuit$ Mauricio: [#1]}}

\begin{document}

\author{Daniel De Roux}
\address{
Departamento de matem\'aticas\\
Universidad de los Andes\\
Carrera $1^{\rm ra}\#18A-12$\\ 
Bogot\'a, Colombia
}
\email{d.de1033@uniandes.edu.co}

\author{Mauricio Velasco}
\address{
Departamento de matem\'aticas\\
Universidad de los Andes\\
Carrera $1^{\rm ra}\#18A-12$\\ 
Bogot\'a, Colombia
}
\email{mvelasco@uniandes.edu.co}

\subjclass[2000]{Primary 15A29 % Inverse problems
Secondary 15B52,52A22} % Random matrices, Random Convex sets and integral geometry
\keywords{Compressed sensing, truncated moment problems, Kostlan-Shub-Smale polynomials}

\begin{abstract} We study the problem of learning the cluster structure of a random graph $\grG$ from an independent sample. We propose a Wasserstein robust formulation of this optimization problem and prove that it can be reformulated as a tractable convex optimization problem. We give theoretical exact recovery guarantees for this problem when the Wasserstein metric is induced by the nuclear norm and $\grG$ is distributed according to the stochastic block model. Finally we present our Julia implementation of the proposed algorithm and show its numerical performance on synthetic data.
\end{abstract} 

\title{Graph clustering and the nuclear Wasserstein metric.}
\maketitle

\section{Introduction}


Let $\grG$ be a random graph with $n$ vertices. By a deterministic summary of $\grG$ we mean a (deterministic) graph $H^*$ which, on average, differs from $\grG$ by as few edges as possible. In this article we study the problem of finding deterministic summaries {\it from an independent sample} of $\grG$ of size $N$. More precisely we address the following problem:

\begin{problem}\label{Prob} Given adjacency matrices $B_1,\dots, B_N$ of an independent sample of $\grG$ find a symmetric matrix $A^*$ in $\argmin_A \EE_{B\sim \grG}[\|A-B\|_1]$. 
\end{problem}

Special cases of this problem arise in cluster detection and in data summarization, both heavily studied in the literature (see Section~\ref{Sec: PreviousWork} for details).

A possible approach to problem~\ref{Prob} is to use the samples to construct the empirical measure $\hat{\mu}:=\sum_{i=1}^N \frac{1}{N}\delta_{B_i}$ as an approximation of the distribution of $\grG$ and to find a minimizer $A$ of the resulting empirical risk
\[ \EE_{B\sim \mu}[\|A-B\|_1]=\frac{1}{N}\sum_{i=1}^N \|A-B_i\|_1.\]

This approach is consistent and will lead to an optimal solution as the sample size $N\rightarrow \infty$. However, when the sample size $N$ is not sufficienly large (typicaly one has few samples of very large graphs) for $\hat{\mu}$ to be a good approximation for the distribution of $\grG$ this approach leads to overfitting. To mitigate this problem we propose a robust version of Problem~\ref{Prob}. In the robust version one aims to minimize the worst-case risk when the distribution of $B$ is allowed to vary in a ball $\mathcal{N}_{\delta}(\hat{\mu})$ of radius $\delta>0$ centered at the empirical measure $\hat{\mu}$ in a suitable metric, leading to 

\begin{problem}\label{ProbRobusto} Given adjacency matrices $B_1,\dots, B_N$ of an independent sample of $\grG$ find a symmetric matrix $\overline{A}$ which minimizes the robust worst-case risk 
\[R_{\delta}(A):=\left(\sup_{\nu\in \mathcal{N}_{\delta}(\hat{\mu})} \EE_{B\sim G}[\|A-B\|_1]\right).\] 
\end{problem}
The robust worst-case risk is obviously dependent on the chosen metric among probability distributions. In this article we will use the Wasserstein metric $W_{\|\bullet\|}$ induced by a norm $\|\bullet\|$ on the space $K$ of symmetric matrices with entries in $[0,1]$. More precisely, if $\nu_1,\nu_2$ are probability measures on $K$ and $\Pi(\nu_1,\nu_2)$ is the set of random variables $(Z_1,Z_2)$ taking values in $K\times K$  with $Z_i\sim \nu_i$ then the Wasserstein distance between $\nu_1$ and $\nu_2$ is given by 
\[ W_{\|\bullet\|}(\nu_1,\nu_2):=\inf_{(Z_1,Z_2)\in \Pi(\nu_1,\nu_2)} \EE[\|Z_1-Z_2\|].\]

The seminal work of Esfahani and Kuhn~\cite{EsfahaniKuhn} shows that robust formulations defined using the Wasserstein metric often lead to tractable convex optimization problems. Our first result is that this is also true for Problem~\ref{ProbRobusto} when the Wasserstein metric is induced by any semidefinitely representable norm.

\begin{theorem}\label{thm: finiteConvex} Let $K$ be the set of $n\times n$ symmetric matrices with entries in $[0,1]$, let $\|\bullet\|$ be a norm on $K$ and let $\delta>0$. Problem~\ref{ProbRobusto} is equivalent to:
\[\min_{(A,s_1,\dots, s_N,\lambda)\in \mathcal{T}}\left( \lambda\delta +\frac{1}{N}\sum_{i=1}^N s_i + \frac{1}{N}\sum_{i=1}^N \|A-B_i\|_1\right)\]
where $\mathcal{T}$ is the set of $(A,\vec{s},\lambda)\in K\times \RR^n\times \RR$  satisfying the inequalities $\lambda\geq 0$ and $\eta_{E, i}(\lambda)\leq s_i$ as $i=1,\dots, N$ and $E$ ranges over the set of all $\{0,1\}$-symmetric matrices, where
\[\eta_{E, i}(\lambda):= \sup_{Y\in K} \left(\langle E, A-Y\rangle -\lambda \|B_i-Y\|\right)\]
\end{theorem}

\ddr{en la restriccion para $\eta$  no es $\eta_{E, i}(\lambda):= \sup_{Y\in K} \left(\langle E, B_i-Y\rangle -\lambda \|B_i-Y\|\right)$ ? eso es lo que logre probar... sin embargo puede que esté bien y a mi no me salio la prueba.}
The reformulation in Theorem~\ref{thm: finiteConvex} makes our problem into a finite-dimensional convex optimization problem which unfortunately contains  an exponential number of constraints. To address this problem we introduce:
\begin{enumerate}
\item A more tractable simplification which agrees with the original problem whenever the optimal $\overline{A}$ occurs at a matrix with entries in $\{0,1\}$.
\item A relaxation of $(1)$ which takes the form of a regularized empirical risk minimization problem.
\end{enumerate}
leading to practical algorithms for graph summarization. More specifically, we prove the following 

\begin{theorem}\label{Thm: tractable}If $A$ is a symmetric $\{0,1\}$-matrix and $\delta>0$ then:
\begin{enumerate}
\item{  The following equality holds:
\[
R_{\delta}(A)=\min_{\mathcal{H}} \left(\lambda\delta +\frac{1}{N}\sum_{i=1}^N s_i+\frac{1}{N}\sum_{i=1}^N\|A-B_i\|_1\right)
\]
where $\mathcal{H}$ is the set of $(\lambda, s_1,\dots, s_N,\Lambda, W)\in \RR\times\RR^N\times \RR^{n\times n} \times \RR^{n\times n}$ \mv{$\Lambda$ y $W$ son sim\'etricas verdad? asi que deberia ser $K$} \ddr{$\Lambda$ si debe ser simetrica, mientras que la unica restriccion de W es que su normal dual sea menor a $\lambda$. } which satisfy the inequalities:
\[
\begin{array}{l}
\|W\|_{*}\leq \lambda\\
\Lambda \geq 0\\
2A-11^t-W + \Lambda \geq 0\\
\|\Lambda\|_{1}\leq s_i-\langle 2A-11^t-W, B_i\rangle\text{ for $i=1,\dots, N$}
\end{array}
\]}
\item The following inequality holds:
\[R_{\delta}(A)\leq \frac{1}{N}\sum_{i=1}^N\|A-B_i\|_1+ \delta\|2A-11^t\|_*.\]
\end{enumerate}
Moreover, if $\|\cdot\|$ is a semidefinitely representable function then both $(1)$ and the minimization of the right hand side of $(2)$ are semidefinite programming problems.
\end{theorem}

Solving the optimization problems in Theorem~\ref{Thm: tractable} leads to a new algorithm for estimating deterministic graph summaries which we call {\it Wasserstein robust graph summarization}. We carried out extensive numerical experiments applying this algorithm to a variety of graphs $G$ distributed according to the stochastic block model and observed that the regularized problem with the Wasserstein metric {\it induced by the spectral norm} was able to recover the correct cluster structure using only very few samples outperforming all others. This leads us to propose the {\it Wasserstein robust nuclear norm summarization problem}, given by

\begin{equation}\label{WRNNS}
\min_{A\in K} \left(\frac{1}{N}\sum_{i=1}^N\|A-B_i\|_1+ \delta\|2A-11^t\|_*\right) 
\end{equation}

where the nuclear norm $\|B\|_*$ of a matrix $B$ is the dual of the spectral norm and equals the sum of its singular values.

A central result of this article is an exact recovery guarantee for this algorithm. 
More precisely, we show that exact recovery occurs for suitable $\delta>0$ with overwhelming probability on samples distributed according to the stochastic block model, explaining the good practical performance of Wasserstein nuclear norm summarization in cluster detection. In order to describe our exactness guarantee we need to establish notation for the parameters of the stochastic block model. Recall that a random graph $\grG$ has distribution given by the stochastic block model in $n$ vertices if there is a partition of $[n]$ into disjoint subsets $C_1,\dots, C_k$, real numbers $0\leq q<\frac{1}{2}<p_1,\dots, p_k\leq 1$ and edges are added independently with probability $p_{ij}$ of joining vertices $i,j$ given by
\[p_{ij}:=\begin{cases}
p_t\text{, if $\{i,j\}\subseteq C_t$}\\
q\text{, else.}
\end{cases}
\] 
Such a random graph $\grG$ has a unique deterministic summary $A^*$ obtained by putting edges only between vertices belonging to the same cluster.

\begin{theorem}\label{thm. performance} Suppose $B_1,\dots, B_N$ are independent and are distributed according to $\grG$. If $\alpha = \min (|p_t-\frac{1}{2}|, |q-\frac{1}{2}|)$ and $\delta^*$ is the maximum of $a(\delta):=\frac{\delta\left(\alpha-\frac{\delta}{n}\right)^2}{\left(1+\frac{2\delta}{n}\right)}$ in $[0,\alpha n]$ then the probability that $A^*$ is not the unique minimizer of~(\ref{WRNNS}) is bounded above by
\[ \exp\left(-\frac{2N(n-1)^3(\alpha n-\delta^*)^2}{n}\right) + e^{-N\left(\delta^* a(\delta^*)\right)}\prod_{i\neq j}\left(1+ (e^{-4a}\widetilde{p_{ij}}+\widetilde{q_{ij}})^{\frac{N}{2}}\right).\] 
which decreases exponentially with the sample size $N$.
\end{theorem}

The key point of the proof of Theorem~\ref{thm. performance}, discussed at length in Section~\ref{Clustering}, is that the subdifferential of the regularization term at $A^*$ is sufficiently rich so as to contain enough transportation matrices. This gives a geometric explanation for why the nuclear norm is a good regularizer for graph summarization problems.
 
Finally we focus on the practical performance of the proposed algorithm. Solving the optimization problems appearing in Theorem~\ref{Thm: tractable} typically require solving large semidefinite programs which are beyond the  capacity of standard off-the-shelf software even for relatively small graphs (of say $40$ vertices with $N=4$). One possible reason is that off-the-shelf solvers often use interior point methods, which are highly accurate but often do not scale well. A better alternative, especially well suited for Wasserstein robust nuclear norm summarization is to use first order numerical optimization methods such as the alternating direction method of multipliers (ADMM). In Section~\ref{Numerics} we adapt the ADMM algorithm to our regularized problem and present our open source Julia implementation. This implementation can run the Wasserstein nuclear norm summarization algorithm in graphs of up to 10000 vertices and $N\leq 10$ on a common laptop. 

\subsection{Relation to previous work.}
\label{Sec: PreviousWork}

Understanding structural properties of graphs is a problem of high interest since graphs appear naturally in many areas. The problem of finding communities in a graph is one of the most studied tasks in machine learning, with  applications to Biology \cite{cabreros2016detecting, cline2007integration,xu2002clustering}, social data understanding \cite{domingos2001mining,mishra2007clustering,newman2002random} and other general machine learning tasks such as natural language processing \cite{collobert2011natural,ratinov2009design}. One of the reasons of the importance of this problem is that it allows researchers to organize datasets into sets of similar observations, an then apply learning algorithms on each of the sets.  
Most theoretical results in the community detection problem occur in the context of generative random graph models. Of these the most studied is the stochastic block model~\cite{abbe2017community} which is particularly important since in it communities are defined a priori and thus there is a formally specified underlying true cluster structure with which the output of algorithms can be compared.

There are three main approaches for the problem of community detection on a (single) graph: 

\begin{enumerate}
\item Spectral clustering algorithms, widely used in the detection of sparse communities \cite{bordenave2015non, chin2015stochastic,krzakala2013spectral, massoulie2014community}. 

\item SDP approaches based on the seminal work of Goemans and Williamson \cite{goemans1995improved}, \cite{abbe2016exact,guedon2016community, montanari2016semidefinite}.

\item Decompositions of the adjacency matrix of the graph into a matrix of low rank and a sparse (noise) matrix, using a convex relaxation based on the nuclear norm \cite{candes2011robust,chandrasekaran2011rank,candes2009exact}. These methods have led to several convex optimization algorithms to find communities in graphs \cite{ames2011nuclear,vinayak2014sharp,chen2012clustering,chen2014clustering,oymak2011finding,ailon2013breaking}.

\end{enumerate}

The results in this article differ from previous work in two main accounts:

\begin{enumerate}
\item Our methods apply to the problem of learning from {\it several} samples while, to the best of our knowledge, the above algorithms must be applied apply to a single graph. Practicioners resort to several ``graph combination" methods to transform datasets consisting of several samples into one. Our methods do not require this additional pre-processing step.

\item Formulating learning problems as Wasserstein robust optimization problems leads to regularization algorithms in a principled way. Even in the single-sample case this leads to improvements in performance when compared for instance with the sparse+low-rank approach (see Section~\ref{Numerics}).
\end{enumerate}


\subsection{Acknowledgments.}
We wish to thank Fabrice Gamboa, Mauricio Junca and Thierry Klein for useful conversations during the completion of this project. D. De Roux was partially supported by a grant from Facultad de Ciencias, Universidad de los Andes and by CAOBA. M. Velasco was partially supported by research funds from Universidad de los Andes. 


\section{Robust learning of deterministic summaries}
Let $[n]:=\{1,2,\dots, n\}$. By a graph $G$ on $[n]$ we mean a finite loopless undirected graph with vertex set $[n]$. The adjacency matrix of such a graph is a symmetric $n\times n$ matrix with entries in $\{0,1\}$ and ones in positions $(i,j)$ whenever vertices $i$ and $j$ are adjacent. Let $\mathcal{A}_n$ be the set of adjacency matrices of graphs on $[n]$ (i.e. $\{0,1\}$-matrices of size $n\times n$ which are symmetric and have zeroes in the diagonal). Throughout the article we will use graphs and their adjacency matrices interchangeably. By a random graph $\grG$ on $[n]$ we mean a random variable $B$ taking values in $\mathcal{A}_n$.

\begin{definition} A {\it deterministic summary} of a random graph $\grG$ on $[n]$ is a graph $A^*$ with $A^*\in \argmin_{A\in\mathcal{A}_n} \EE[\|A-\grG\|_1]$.
\end{definition}
Here the norm $\|\bullet\|_1$ means the sum of absolute values of all entries of the matrix. Therefore the deterministic summary $A^*$ is a graph which, on average, differs from $\grG$ by the smallest possible number of edges. If the distribution of $\grG$ is known it is easy to find a deterministic summary (which is often unique). More precisely,

\begin{lemma}\label{lem: withDist} If $\PP\{(i,j)\in \grG\}\neq \frac{1}{2}$ for all $i,j\in [n]$ then the unique deterministic summary of $\grG$ is given by $A^*$ with
\[A^*_{ij}=
\begin{cases}
1\text{, if $\PP\{(i,j)\in \grG\}> \PP\{(i,j)\not\in \grG\}$}\\
0\text{, if $\PP\{(i,j)\in \grG\}< \PP\{(i,j)\not\in \grG\}$}
\end{cases}\]
\end{lemma}
\begin{proof} If $A\in \mathcal{A}_n$ then the term coming from entry $(i,j)$ in $\EE[\|A-\grG\|_1]$ is given by $|A_{ij}-1|p+|A_{ij}|q$ where $p$ (resp $q$) is the probability that $(i,j)$ is (resp. is not) an edge of $\grG$. This quantity is greater than $\min(p,q)$ and equality is achieved, for all $i,j$ with $A=A^*$. 
\end{proof}

Motivated by the previous Lemma we define a cluster structure on a random graph.

\begin{definition} A random graph $\grG$ has a {\it cluster structure} if it has a unique deterministic summary $A^*$ and the corresponding graph is a disjoint union of cliques. We call these cliques the clusters of $\grG$.
\end{definition}

The main problem that we address in this article is that of {\it learning} deterministic summaries (an in particular the problem of learning cluster structures on random graphs who have them). By this we mean that our only knowledge about the distribution of the random graph $\grG$ is encoded in an independent sample $B_1,\dots, B_N$ of adjacency matrices with the same distribution as $\grG$, leading to Problem~\ref{Prob} in the Introduction. 



Given the sample, define the empirical measure $\hat{\mu}:=\frac{1}{N}\sum_{j=1}^N \delta_{B_j}$ as a sum of Dirac delta measures at the sample points. As the number of sample points increases the measure $\hat{\mu}$ converges to the distribution of $\grG$ and it is therefore reasonable to try to minimize the objective function in Problem~\ref{Prob} with respect to the measure $\hat{\mu}$ instead of $\grG$, that is by finding a minimizer of the empirical risk
\[\overline{A}\in\argmin _{A\in K} \EE_{Z\sim \hat{\mu}}[\|A-Z\|]\]
Arguing as in Lemma~\ref{lem: withDist} it is immediate that a (generally unique) minimizer $\overline{A}$ is given by counting edge frequencies, that is
\[ 
\overline{A}_{ij}:=\begin{cases}
1\text{, if $|\{t: B^{(t)}_{ij}=1\}|>|\{t: B^{(t)}_{ij}=0\}|$}\\
0\text{, if $|\{t: B^{(t)}_{ij}=1\}|<|\{t: B^{(t)}_{ij}=0\}|$}
\end{cases}
\]

The empirical risk minimization approach has lots of advantages, it is easy to implement, scales very well and is guaranteed to be consistent (in the sense that $\overline{A}_{ij}\rightarrow \overline{A}$ as the number of samples $N\rightarrow \infty$). However it also suffers from some potential drawbacks:

\begin{enumerate}
\item If the sample size $N$ is small then the empirical measure $\hat\mu$ could be very far from the distribution of $\grG$.
\item The estimation of $\overline{A}$ is done independently edge by edge and in particular it does not use any global information, for instance the existence of a cluster structure as part of the estimation process.
\end{enumerate}

To mitigate these problems we will use a robust formulation. To this end let $\|\bullet\|$ be a norm in the space of symmetric $n\times n$ matrices and let $W_{\|\bullet\|}$ be the Wasserstein distance induced by $\|\bullet\|$ on probability distributions taking values on $K$. For a real number $\delta\geq 0$ let $\mathcal{N}_{\delta}(\hat{\mu})$ be the (closed) ball of radius $\delta$ centered at $\hat{\mu}$. We would like to solve the following


\begin{problem*}\label{ProbRobusto2} Given adjacency matrices $B_1,\dots, B_N$ of an independent sample of $\grG$ find a symmetric matrix $\overline{A}$ which minimizes the robust worst-case risk 
\[R_{\delta}(A):=\left(\sup_{\nu\in \mathcal{N}_{\delta}(\hat{\mu})} \EE_{B\sim G}[\|A-B\|_1]\right).\] 
\end{problem*}

Our first result reformulates the robust optimization above as a finite convex optimization problem.
\begin{proof}[Proof of Theorem~\ref{thm: finiteConvex}]


Denote the set of symmetric matrices with entries in $\{-1,1\}$ by $L$.
Let $\mathcal{M}(K\times K,\hat{\mu})$ be the set of probability measures in $K\times K$ whose marginal distribution in the first component is given by $\hat{\mu}$. We will refer to this set as $\mathcal{M}$ from now on. Every such measure is of the form 
\[\mu = \sum_{i=1}^N c_i \delta_{B_i}\otimes \mathbb{Q}_i\]
where $\mathbb{Q}_i$ is a probability measure on $K$. These measures act on functions on two sets of variables corresponding to each copy of $K$. More concretely
\[\int_{K\times K} f(x,y) d\mu = \sum_{i=1}^ N c_i \int_K f(B_i,y)d\mathbb{Q}_i(y).\]




\[R_{\delta}(A)=\left(\sup_{\nu\in \mathcal{N}_{\delta}(\hat{\mu})} \EE_{B\sim G}[\|A-B\|_1]\right) = \sup_{\nu\in \mathcal{N}_{\delta}(\hat{\mu})} \int \|A-Y\|_1 d\nu \]

since $ \nu\in \mathcal{N}_{\delta}(\hat{\mu}) $ we can rewrite the right term to obtain that

\begin{equation}\label{rdelta}
\begin{aligned}
R_{\delta}(A) &= \sup_{\mu \in \mathcal{M}} \inf_{\lambda \geq 0} \int_{K \times K}\|A-Y\|_1 d\mu + \lambda \left (  \delta - \int_{K \times K} \|X-Y\|d\mu.\right )\\
&=  \sup_{\mu \in \mathcal{M}} \inf_{\lambda \geq 0} \int_{K \times K}\|A-Y\|_1 d\mu + \lambda \delta - \lambda \frac{1}{N}\sum_{i=1}^{N} \int_K \|B_i-Y\|d\mathbb{Q}_i(y)
\end{aligned}
\end{equation}
\ddr{aqui hay un detalle de notacion.  para la segunda integral ya use la forma de mu, pero para escribimos que el supremo se toma sobre mu en M y no sobre las medidas Q.}
Now, note that
\[
\|A-Y\|_1 = \sup_{E \in L} \left \langle A-Y,E\right \rangle.
\]



and using bilinearity, we have that 
\[
\sup_{E \in L} \left \langle A-Y,E\right \rangle = \sup_{E \in L} \left ( \left \langle A-X,E\right \rangle +  \left \langle X-Y,E\right \rangle \right ).
\]
where $X \sim \hat{\mu}$.
Given that we know how $\mu \in \mathcal{M}$ acts, we can reformulate  the first integral as follows:
\[
\begin{aligned}
\int_{K \times K} \|A-Y\|_1 d\mu &= \int_{K \times K} \sup_{E \in L} \left ( \left \langle A-X,E\right \rangle +  \left \langle X-Y,E\right \rangle \right ) d\mu \\
& = \sup_{E \in L} \frac{1}{N} \sum_{i=1}^n   \langle A-B_i,E  \rangle +\frac{1}{N}\sum_{i=1}^N \int_K \sup_{E \in L} \langle B_i-Y,E\rangle d \mathbb{Q}_i \\
& = \frac{1}{N} \sum_{i=1}^N \|A-B_i\|_1 +\frac{1}{N}\sum_{i=1}^N \int_K \sup_{E \in L} \langle B_i-Y,E\rangle d \mathbb{Q}_i 
\end{aligned}
\]
Replacing this equation in \ref{rdelta} and regrouping the terms we get that:
\[
R_{\delta}(A) = \sup_{\mathbb{Q}_i} \inf_{\lambda \geq 0} \lambda \delta + \frac{1}{N} \sum_{i=1}^N \|A-B_i\|_1  +\frac{1}{N}\sum_{i=1}^N \left ( \int_K \sup_{E \in L}(\langle B_i-Y,E \rangle) -\lambda \|B_i-Y\|   d \mathbb{Q}_i   \right )
\]
Exchanging sup and inf and noting that the $\mathbb{Q}_i$ are arbitrary measures, we 
obtain that this quantity is bounded above \ddr{equals?}

\[
\inf_{\lambda \geq 0} \lambda \delta + \frac{1}{N} \sum_{i=1}^N \|A-B_i\|_1  +\frac{1}{N}\sum_{i=1}^N \sup_{Y \in K} \left ( \sup_{E \in L}(\langle B_i-Y,E \rangle) -\lambda \|B_i-Y\|   \right )
\]

Finally,we put the supremum into the constraints obtaining the equivalent problem

\[ \min_{A} R_{\delta}(A)=\min_{(A,s_1,\dots, s_N,\lambda)\in \mathcal{T}} \lambda \delta + \frac{1}{N} \sum_{i=1}^N \|A-B_i\|_1  +\frac{1}{N}\sum_{i=1}^N s_i
\]
where $\mathcal{T}$ is the set of $(A,\vec{s},\lambda)\in K\times \RR^n\times \RR$  satisfying the inequalities $\lambda\geq 0$ and $\eta_{E, i}(\lambda)\leq s_i$ as $i=1,\dots, N$ and $E$ ranges over the set of all $\{-1,1\}$-symmetric matrices, where
\[\eta_{E, i}(\lambda):= \sup_{Y\in K} \left(\langle E, B_i-Y\rangle -\lambda \|B_i-Y\|\right)\]
\end{proof}

Our second results gives a simplification that agrees with the original problem when the optimal of problem \ref{ProbRobusto} occurs at matrix with entries in $\{0,1\}$.
The proof is similiar to the proof of \ref{thm: finiteConvex}.


\begin{proof}[Proof of Theorem~\ref{Thm: tractable}]
Since $A$ has entries in $\{0,1\}$ and $B$ has entries in $K$ the equality $\|A-B\|_1=\langle 2A-11^t, A-B\rangle=\|A-\hat{B}\|_1+\langle 2A-11^t, \hat{B}-B\rangle$  where $\hat{B}$ is distributed according to the empirical measure holds. Taking expected values and suprema we see that
\[\alpha=\sum_{i=1}^N c_i\|A-B_i\|_1+\sup_{\nu\in B_{\delta}(\hat{\mathbb{P}_N})}\EE\left[\langle 2A-11^t, \hat{B}-B\rangle\right].\] 
Recall that if $\mu \in \mathcal{M}$, then $\mu$ is of the form 
\[\mu = \sum_{i=1}^N c_i \delta_{B_i}\otimes \mathbb{Q}_i\]

where $\mathbb{Q}_i$ is a probability measure on $K$.

This allows us to rewrite
\[\sup_{\nu\in B_{\delta}(\hat{\mu})}\EE\left[\langle 2A-11^t, \hat{B}-B\rangle\right] = 
\sup_{\mu\in \mathcal{M}} \inf_{\lambda\geq 0} \int_{K\times K}\langle 2A-11^t, X-Y\rangle d\mu +\lambda\left(\delta -\int_{K\times K} \|X-Y\|_*d\mu\right)\]
Exchanging sup and inf we obtain that this quantity is bounded above \mv{equals?}
\[\inf_{\lambda\geq 0} \left(\lambda \delta +\sup_{\mu\in \mathcal{M}}\int_{K\times K} \langle 2A-11^t, X-Y\rangle -\lambda \|X-Y\|_*d\mu\right)\]
Using the fact that $\mu = \sum_{i=1}^N c_i \delta_{B_i}\otimes \mathbb{Q}_i$ this quantity equals
                                                                                                                                                                                                                                                                               \[\inf_{\lambda\geq 0} \left(\lambda\delta + \sum_{i=1}^N c_i \sup_{\mathbb{Q}_i} \int_{K} \langle 2A-11^t, B_i-Y\rangle -\lambda \|B_i-Y\|_*d\mathbb{Q}_i\right)\]
Since $\mathbb{Q}_i$ are arbitrary probability measures this quantity equals
\[\inf_{\lambda\geq 0} \lambda \delta +\sum_{i=1}^N c_i \sup_{Y\in K} \left(\langle 2A-11^t, B_i-Y\rangle -\lambda \|B_i-Y\|_*\right)\]

Next we put the supremum into the constraints obtaining the equivalent problem
\[ \inf_{(\lambda,s_1,\dots, s_N)} \lambda\delta +\sum c_is_i\]
over the set given by the inequalities $\lambda\geq 0$ and for which the following inequalities in $\lambda$ are satisfied for $i=1,\dots, N$.
\[\sup_{Y\in K} \left(\langle 2A-11^t, B_i-Y\rangle -\lambda \|B_i-Y\|_*\right)\leq s_i\]


Next we compute the supremum in the constraints using strong duality \mv{Check hypothesis for strong duality}. We use duality because we are aiming to bound  the supremum above by $s_i$ and for this it suffices to exhibit a point in the dual whose objective function is bounded above by $s_i$. Denote the dual norm of $\|\bullet\|$ by $\|\bullet\|_{d^*}$.



\[
\begin{aligned}
\sup_{Y\in K} \left(\langle 2A-11^t, B_i-Y\rangle -\lambda \|B_i-Y\|_*\right)&=
\sup_{Y\in K} \inf_{ \|W\|_{d*}\leq \lambda} \langle 2A-11^t-W, B_i-Y\rangle \\ 
&=\inf_{ \|W\|_{d*}\leq \lambda}\sup_{Y\in K}\langle 2A-11^t-W, B_i-Y\rangle 
\end{aligned}
%=\inf_{W: \|W\|_*\leq \lambda} \langle 2A-11^t-W, B_i\rangle+ \sup_{Y\in K} \langle D,-Y \rangle.
\]
By the comment above, we wish to bound the last term by $S_i$, so it is enough to exhibit a $W$ such that $W: ||W||_{d*} \leq \lambda$ such that
\[
\sup_{Y\in K} \langle 2A-11^t-W, B_i-Y\rangle \leq S_i.
\]
For the sake of notation, let 
$D= 2A-11^t-W$.
The last inequality then becomes
\[
\langle D,B_i\rangle + \sup_{Y \in K}\langle D,-Y \rangle \leq S_i.
\]
which is equivalent to 
\[
 \sup_{Y \in K}\langle -D,Y \rangle \leq S_i-\langle D,B_i\rangle.
\]
We will now compute 
\[\sup_{Y \in K}\langle -D,Y \rangle.\]
$K$ is the hipercube, which gives the restrictions $0 \leq Y \leq 11^t$ and $Y=Y^t$. 
The lagragian is: 
\[
L = \langle D,-Y \rangle + \langle \Lambda,11^t-Y \rangle + \langle \eta,Y \rangle + \langle \Omega,Y-Y^t \rangle
= \langle D+\Lambda -\eta,-Y \rangle + \langle \Lambda,11^t\rangle + \langle \Omega, Y-Y^t\rangle\]
with $\Lambda,\eta \geq 0$.
It follows that
\[
\sup_{Y \in K}\langle -D,Y \rangle = \sup_{Y \in K} \inf_{\Lambda,\eta \geq 0} L = \begin{cases}
\langle 2A-11^t-W,-Y \rangle \ \text{if } Y \in K\\
- \ \infty \text{ in the other case.}
\end{cases}
\]
Exchanging infimum and supremum, we obtain  
\[
\sup_{Y \in K}\langle -D,Y \rangle = \inf_{\Lambda,\eta \geq 0}\sup_{Y \in K}L= \inf_{\Lambda,\eta \geq 0} 
\begin{cases}
\langle \Lambda, 11^t \rangle \ if \ D+\Lambda + \eta = 0 \ and \ \Omega =0.\rangle \\
\infty \text{ in the other case.}
\end{cases}
\]
Finally, this problem reduces to, 
\[
\inf_{\Lambda \geq0} \|\Lambda\|_1
\]
Subject to 
\[
\Lambda, \ D+\Lambda \geq 0.
\]




\end{proof}

\section{Clustering in the stochastic block model and the Wasserstein nuclear norm.}
\label{Clustering}

Suppose $B$ is a random (undirected loopless) graph on $n$ vertices generated by the stochastic block model. This means that we fix a set partition $C_1,\dots, C_l$ of $[n]$ into sets we call clusters and real numbers $0\leq p_i,\bar{p}\leq 1$ for $i=1,\dots, l$. The edges of $B$ are independent random variables and an edge joins vertices $i,j$ with probability $p_t$ if $\{i,j\}\subseteq C_t$ for some cluster $C_t$ and with probability $\bar{p}$ if $\{i,j\}$ is not contained in any $C_t$. We let $O\subseteq [n]\times [n]$ be the set of pairs of vertices which are not simultaneously contained in any cluster. 

For an integer $N$ let $B_1,\dots, B_N$ be an independent sample of $N$ graphs with the distribution of $B$. Let $A^*$ be the $n\times n$ matrix with entries in $\{0,1\}$ which captures the underlying cluster structure, namely $A^*_{ij}=1$ iff there is a cluster $C_t$ which contains both $ij$.
\ddr{or if i=j. This is important as proofs are clearer if we consider that the entries in the diagonal are not in I or in O.}
In this section we study the probability, as a function of $\delta$ that the optimization problem $\min_A\Delta(A)$
\[\Delta(A)= \delta\|2A-11^t\|_{*}+\frac{1}{N}\sum_{k=1}^N\|A-B_k\|_1\] 
has the correct cluster structure $A^*$ as a minimizer. For vertices $i,j\in [n]$ define $n_1(ij)$ (resp. $n_0(ij)$) the random variables which count the number of times that a given pair is (resp is not) an edge of some $B_j$, $j=1,\dots N$. Note that the $n_q(ij)$ for $q=0,1$ are binomial random variables.  

\begin{lemma} The following statements hold:
\label{lem: subdiff}
\begin{enumerate}
\item The subdifferential of $\frac{1}{N}\sum_{k=1}^N\|A-B_k\|_1$ at $A^*$ is the set of symmetric matrices $C$ satisfying the inequalities
\[ \frac{n_0(ij)-n_1(ij)}{N}\leq C_{ij}\leq 1 \text{, if $\{i,j\}\subseteq C_t$ for some $t$,}\]
\[-1\leq C_{ij} \leq \frac{n_0(ij)-n_1(ij)}{N} \text{ if $\{i,j\}$ does not belong to any cluster and } \]
\[-N \leq C_{ii} \leq N \text{ for all }i.\]

\item The subdifferential of $\delta\|2A-11^t\|_{*}$ at $A^*$ is given by the set of symmetric matrices of the form $2\delta C$ where $C$ has spectral norm $\|C\|\leq 1$ and satisfies $\langle C, 2A-11^t\rangle = n$.

\end{enumerate}

\end{lemma}
\begin{proof} $(1)$ Since the subdifferential is additive it suffices to understand the subdifferential of the absolute value. If $i,j\in C_t$ then $A^*_{ij}=1$ and the entry $ij$ of the subdifferential of the sum at $A^*$ is $[-1,1]$ for each $B_i$ containing the edge and it is $1$ for each $B_i$ for which $(ij)$ is not an edge. \ddr{vale la pena mencionar el caso i=j o es obvio la propiedad en la diagonal?} If $i,j$ is not contained in any cluster then $A^*_{ij}=0$ and the entry $ij$ of the subdifferential of the sum at $A^*$ is $-1$ for each $B_i$ which contains the edge $ij$ and $[-1,1]$ for each $B_i$ which does not, proving the claim. $(2)$ It is easy to prove that the subdifferential of any norm $\|\bullet\|$ at a point $X$ is given by those $C$ for which the dual norm $\|C\|_*\leq 1$ and $\langle C,X\rangle =\|X\|$. Claim $(2)$ follows because $\|2A-11^t\|=Tr(2A-11^t)=n$ where the first equality holds since $2A-11^t$ is positive semidefinite.\mv{Esto es obvio con solo dos clusters (la matriz es $uu^t$ donde $u$ es el vector con $1$'s en un cluster y $-1$'s en el complemento pero hay que demostrarlo para tres o mas)}.
\end{proof}

%\begin{lemma} If $\delta=0$ then 
%\[\PP\{A^*\in\argmin\Delta \} =\prod_{ij\in O} \PP\{n_1(ij)\leq n_0(ij)\} \prod_{t=1}^k \left(\prod_{(ij)\in C_t}\PP\{n_0(ij)\leq n_1(ij)\}\right)\]
%where $\PP\{n_0(ij)\leq n_1(ij)\}$ is given by the following formula 
%\mv{Ejercicio para Daniel: Encontrar una f\'ormula, es la probabilidad de que haya mas caras que sellos en $N$ lanzamientos de una moneda trucada donde las probabilidades de la moneda dependen s\'olo de la arista $ij$}.
%\end{lemma}

%\begin{proof} The matrix $A^*$ is a minimizer of the above convex function if and only if its subdifferential at $A^*$ contains the matrix $0$. By part $(1)$ of the previous Lemma this occurs if and only if $\frac{n_0(ij)-n_1(ij)}{N}\leq 0$ for $(ij)$ in a cluster and 
%$\frac{n_0(ij)-n_1(ij)}{N}\geq 0$ for $(ij)$ in $O$. Independence of the edges then implies the above formula.  
%\end{proof}
%\begin{remark} Could the set of minimizers be larger? If $A'$ has at least one entry $A_{ij}\in (0,1)$ the corresponding component in the subgradient is the constant $n_0(ij)-n_1(ij)$ and this equals zero with much smaller probability, precisely when both terms equal to $\frac{N}{2}$. In particular it is impossible if $N$ is odd and in this case $A^*$ is the only minimizer.
%\end{remark}

%Next we ask whether it is possible to increase the probability of correct recovery by allowing $\delta>0$. By Lemma~\ref{lem: subdiff} $A^*\in \argmin \Delta$ if and only if there exists $S$ in the subdifferential of $\delta\|2A-11^t\|_*$ at $A=A^*$ such that $-S$ belongs to the subdifferential of $\frac{1}{N}\sum_{k=1}^N\|A-B_k\|_1$ at $A^*$. 

%The most immediate way to do this would be to find an $S$ with $S_{ij}\leq 0$ negative on edges $ij\in C_t$ and $S_{ij}\geq 0$ on edges of $O$. More precisely we would like to find a best such $C$ by solving the optimization problem:

%\[
%\min \left(\sum_{t}\sum_{(ij)\in C_t} C_{ij}\right)-\sum_{(ij)\in O} C_{ij} \text{ s.t. $\|B\|\leq 1$, $\langle C, 2A^*-11^t\rangle = n$} 
%\]

%However it is easy to see that we cannot do this improvement simultaneously in all components, because $n=\langle S, 2A-11^t\rangle = Tr(S)+\sum_{ij \in O^c} S_{ij} -\sum_{ij \in O} S_{ij}\leq n+u$ where $u$ is the objective function in the problem above. We conclude that $u$ must be nonnegative so we cannot improve simultaneously in all directions at once.
In the following section we discuss how tradeoffs between components explain the improved recovery probability induced by the spectral norm.


Let $\Gamma$ be the symmetric matrix with zero diagonal and off-diagonal entries given by
$\Gamma_{ij}=\frac{n_0(ij)-n_1(ij)}{N}$. By Lemma~\ref{lem: subdiff} a symmetric matrix $C$ lies in the subdifferential if and only if it satisfies the inequalities
\[ \Gamma_{ij}\leq C_{ij}\leq 1 \text{, if $\{i,j\}\subseteq C_t$ for some $t$,}\]
\[-1\leq C_{ij} \leq \Gamma_{ij} \text{ if $\{i,j\}$ does not belong to any cluster and } \]
\[-N \leq C_{ii} \leq N \text{ for all }i.\]
To simplify these inequalities we define a linear operator $\widetilde{\bullet}$ on symmetric matrices by the formula
\[ \widetilde{A} = 
\begin{cases}
A_{ij}\text{ if $i=j$ or $ij\in I$ and}\\
-A_{ij}\text{ if $ij\in O$.} 
\end{cases}
\]
in this language $C_{ij}$ belongs to the subdifferential if and only if $\widetilde{\Gamma}_{ij}\leq \widetilde{C}_{ij}$ for $i\neq j$.
The following key result gives sufficient conditions for the true cluster structure $A^*$ to be a minimizer of the proposed optimization problem. In order to describe it we introduce the following notation. 

\begin{definition} Let $\delta$ be a positive real number. For a symmetric matrix $\Gamma$ define the quantities
\[b(\Gamma,\delta):=\sum_{i\neq j} \max\left(\widetilde{\Gamma_{ij}}+\frac{2\delta}{n},0\right)
\text{ and } a(\Gamma,\delta):=\sum_{i\neq j} \max\left(-\widetilde{\Gamma_{ij}}-\frac{2\delta}{n},0\right)
\]\end{definition}
The quantity $b(\Gamma,\delta)$ (resp. $a(\Gamma,\delta)$) measures the total amount by which the matrix $\widetilde{-\frac{2\delta}{n}11^t}$ fails (resp. succeeds) to be in the subdifferential of Lemma~\ref{lem: subdiff} in the sense that it sums over all $ij$ the amount by which the inequalities $\widetilde{\Gamma_{ij}}\leq \frac{2\delta}{n} 11^t$ fail (resp. succeed). The key point of the following Theorem is that if the inequality fails by less than it succeeds then the subdifferential of the spectral norm is sufficiently rich so as to allow us to redistribute these quantities. In this sense the following Theorem explains the success of the spectral norm in cluster recovery algorithms. 


Recall that 
\[\Delta(A)= \delta\|2A-11^t\|_{*}+\frac{1}{N}\sum_{k=1}^N\|A-B_k\|_1\] 
And define functions $f$, $g$:
\[
f:= \frac{1}{N}\sum_{k=1}^N\|A-B_k\|_1, \ \ g:= \delta\|2A-11^t\|_{*}
\]
In this vocabulary, we have that
\begin{equation}\label{SumaSubDif}
\partial(\Delta) = \partial(f)(A^*)+\partial(g)(A^*)
\end{equation}




\begin{theorem}\label{thm: transport} [Alternative proof of theorem 6.2] Assume there are only two clusters. Let $\delta>0$ with $ \left(\frac{\delta}{n}+b(\Gamma,\delta) \right)  < \frac{N}{2}$. If $b(\Gamma,\delta)< \min(\delta, a(\Gamma,\delta))$ then $A^*$ is a minimizer of the optimization problem $\min_A\Delta(A)$.
\end{theorem} 


\begin{proof}
We will show that there exists a matrix $C $ such that 
$-C \in \partial(g)(A^*) $ for which $\widetilde{\Gamma}_{ij}\leq \widetilde{C_{ij}}$ for $i\neq j$. This implies that $C_{ij} \in \partial(f)(A^*)$ and therefore $0=C-C$ belongs to the subdifferential $\partial(\Delta)(A^*)$ and thus $A^*$ is a minimizer of $\Delta(A)$.

Recall that $-C \in  \partial(g)(A^*) $ if and only if 
\[
\left \langle H^*,C \right \rangle = -2\delta n \text{ and } \|C\| \leq 2\delta
\]
where $H^* = 2A-11^t$ and $\|\bullet\|$ is the spectral norm.

Notice that both these conditions are satisfied by setting
$C^0 = -\frac{2\delta}{n}H^*$ as
\[
 \frac{-2\delta}{n}\left \langle H^*,H^* \right \rangle =  \frac{-2\delta}{n}n^2 = -2\delta n \text{ and } \|C^0\|= \frac{2\delta}{n}\|H^*\|= 2\delta.
\]

However this choice of $C^0$ will not, in general, satisfy the inequalities $\widetilde{\Gamma}_{ij}\leq \widetilde{-\frac{2\delta}{n} H^*}_{ij}$ for $i\neq j$. 
Therefore, we will correct our candidate matrix $C^0$ so that it satisfies these inequalities and still belongs to $\partial(g)(A^*)$.

To do this, we will construct a matrix $K$ and add it to $C^0$. Crucially, $K$ will satisfy that $KH^* = H^*K = 0$ so that we can control the spectral norm of $C^0+K$.

Let $i<j$ and 
define the symmetric matrix  $e^{ij}$ as follows:
\begin{equation}
e^{ij}=\begin{cases}
1 \text{ if }(i,j) \in I. \\
-1 \text{ if }(i,j) \in O. \\
-1 \textit{ in the entry } ii \text{ and in } jj. \\
0 \text{ otherwise}. 
\end{cases}
\end{equation}

Observe that for any $(i,j)$ the matrix $e^{ij}$ satisfies the following  properties:

\begin{enumerate}
\item  $H^*e^{ij}=0=e^{ij}H^*$. 
\item The inequality $\|-e^{ij}+e^{st}\|\leq 2$ holds for all $ij$ and $st$. This is immediate noting that the spectral norm is bounded by the product of the induced $1-norm$ and the induced $\infty-$norm. (The inequality is strict only if $|\{i,j\}\cap\{s,t\}|\geq 1$ and in this case it can take values of $\sqrt{3}$ and $0$).  
\end{enumerate}

The idea is to use that $b(\Gamma,\delta) < a(\Gamma,\delta)$ so there exists a way to redistribute the quantity $b(\Gamma,\delta)$ by subtracting it from the $ij$ for which $\frac{2\delta}{n}< \widetilde{\Gamma}_{ij}$  and adding it into those $st$ for which $\widetilde{\Gamma}_{st}\leq \frac{2\delta}{n}$.


Let $U$ be the set of entries $\{i,j\}$ where $\widetilde{\Gamma}_{ij}+ \frac{2\delta}{n} \leq 0$ and $V$ be the  set of entries $\{i,j\}$ where $-(\widetilde{\Gamma}_{ij}+ \frac{2\delta}{n}) > 0$.
Observe that 

\[
b(\Gamma,\delta) = \sum_{ij \in U}( \widetilde{\Gamma}_{ij}+ \frac{2\delta}{n}) \text{ and } a(\Gamma,\delta) =  \sum_{ij \in V} -(\widetilde{\Gamma}_{ij} + \frac{2\delta}{n}).
\]

 By hypothesis $b(\Gamma,\delta)< a(\Gamma,\delta)$.
Let $l_1,..,l_k$ be an enumeration of $V$ were $k$ is it's cardinality. For each entry $ij \in U$, there exists nonnegative coefficients $\gamma^{ij}_{l_1},..,\gamma^{ij}_{l_k}$ such that:

\[
 \widetilde{\Gamma}_{ij}+\frac{2\delta}{n}-\gamma^{ij}_{l_1}-...-\gamma^{ij}_{l_k} < 0. 
\]
and that such that for each  $\gamma_{l_p}$ with $p \in \{1,..,k\}$
\begin{equation}{\label{desA}}
-(\widetilde{\Gamma}_{l_p}+\frac{2\delta}{n})-\sum_{ij \in U}\gamma^{ij}_{l_p}>0.
\end{equation}




For $ij \in U$ define the matrix
\[
W^{ij}:= \gamma^{ij}_{l_1}(\widetilde{e^{ij}-e^{l_1}} )+...+\gamma^{ij}_{l_k}(\widetilde{e^{ij}-e^{l_k}})
\]
Given that $ij \in U$, $l_1,..,l_k \in V$ and the sets $U$ and $V$ are disjoint, the support of $e^{ij}$ is disjoint from the support of any of the matrices $e^{l_1},..,e^{l_k}$ (except probably at the diagonal). In particular, if $ij\in I$ the entry $ij$ of the matrix $W^{ij}$, namely $W^{ij}_{ij}$ is equal to:
\[
\gamma^{ij}_{l_1}(1-0)+...+\gamma^{ij}_{l_k}(1-0)= \gamma^{ij}_{l_1}+...+\gamma^{ij}_{l_k}.
\]
and if  $ij\in O$,
\[
\gamma^{ij}_{l_1}(-1-0)+...+\gamma^{ij}_{l_k}(-1-0)= -\gamma^{ij}_{l_1}-...-\gamma^{ij}_{l_k}.
\]

Define the matrix $C^1$ as:
\[C^1:= C^0 + \sum_{ij \in U} W^{ij} \]
We will now verify that $C^1 \in \partial(f)(A^*)$.
For $ij \in V$, we have by definition of $V$ and by equation \ref{desA} that $\widetilde{\Gamma}_{ij}< \widetilde{C}^1_{ ij}$.  

Let $ij \in I\cap U$. We have that $\Gamma_{ij}+\frac{2\delta}{n}>0$.
Now, the entry of $C^1$ in $ij$ is given by:
\[
C^1_{ij} = -\frac{2\delta}{n}H_{ij}+W^{ij}_{ij} = - \frac{2\delta}{n}+\gamma^{ij}_{l_1}+...+\gamma^{ij}_{l_k}
\]
By the construction of the $\gamma$'s, we have that
\[
\Gamma_{ij}+\frac{2\delta}{n}-\gamma^{ij}_{l_1}-...-\gamma^{ij}_{l_k}<0
\]
it follows that 
\[
\Gamma_{ij}-C^1_{ij}<0
\]
so that
\[
C^1_{ij}>\Gamma_{ij}.
\]
For $ij \in O\cap U$, we have that $\frac{2\delta}{n}-\Gamma_{ij}>0$.
The entry of $C^1$ in $ij$ is given by:
\[
C^1_{ij}=-\frac{2\delta}{n}H_{ij}+W^{ij}_{ij} = \frac{2\delta}{n}-\gamma^{ij}_{l_1}-...-\gamma^{ij}_{l_k}
\]
Since
\[
\frac{2\delta}{n}-\Gamma_{ij}-\gamma^{ij}_{l_1}-...-\gamma^{ij}_{l_k}<0
\]
it follows that
\[
-C_{ij}>-\Gamma_{ij}.
\]

and that $\widetilde{\Gamma}_{ij} < \widetilde{C}^1_{ ij}$ for $i\neq j$. 

It remains to show that the entries of the diagonal of $C^1$ are bounded by $n$. The diagonal entries $\{ss\}$ of $C^1$ are given by
\[
-\frac{2\delta}{n}+ \sum_{ij\in U} W^{ij}_{ss}.
\]


Notice that by the definition of $W^{ij}$, each of the matrices $(\widetilde{e^{ij}-e^{l_p}} )$ has, in the worst case, a $-2$ in the entry $ss$ so the entry $ss$ of $C^1$ is bounded below by 
\[
-\frac{2\delta}{n}+-2 \left(\sum_{ij\in U}\sum_{l_p \in V} \gamma_{l_p}^{ij}\right)
\]
the quantity in the parentheses is all the weight that we have to distribute, i.e $b(\Gamma,\delta)$.
Therefore, 
\[
C^1 \geq -\frac{2\delta}{n}-2(b(\Gamma,\delta)) =-2\left(\frac{\delta}{n}+b(\Gamma,\delta)\right).
\]
By hypothesis, $ \frac{N}{2} > \left(\frac{\delta}{n}+b(\Gamma,\delta) \right) $ so we obtain that
\[
C^1_{ss} > -N.
\]
It is obvious that $N > C^1_{ss}$.
We conclude that $C^1 \in \partial(f)(A^*)$.

For each $ij$, $H^*e^{ij}=0=e^{ij}H^*$ so the equality $\langle H^*, C^1\rangle =\langle H^*,-\frac{2\delta}{n}H^*\rangle = -2\delta n$ holds and moreover

\[ \| C^1\|=\max\left(\left\|-\frac{2\delta}{n}H^*\right\|, \left\|\sum_{ij\in U} W^{ij} \right\|\right).
\]


The operator norm of the first term in the maximum equals $2\delta$ and that of the second term is bounded by $2b(\Gamma,\delta)$ by the triangle inequality and the definition of $b(\Gamma,\delta)$. We conclude that $\|C\|$ is bounded by $2\delta$ because $b(\Gamma,\delta)\leq \delta$. As a result $-C\in \partial\left(g\right)(A^*)$ proving the Theorem. Note that $C^1$ satisfies all the inequalities that define the membership to $\partial(f)(A^*)$ in \ref{lem: subdiff} strictly, so $C^1$ belongs is interior point of $\partial(f)(A^*)$.





\end{proof}






%\begin{theorem}\label{thm: transport} Assume there are only two clusters. If $b(\Gamma,\delta)\leq \min(\delta, a(\Gamma,\delta))$ then $A^*$ is a minimizer of the optimization problem $\min_A\Delta(A)$. 
%\end{theorem}
%\begin{proof} We will show that there exists a matrix $C_{ij}$ such that $-C_{ij}\in \partial \left(\delta\|2A-11^t\|\right)(A^*)$ for which $\widetilde{\Gamma_{ij}}\leq \widetilde{C_{ij}}$ for $i\neq j$. It will then follow that $0=C-C$ belongs to the subdifferential of $\Delta(A)$ at $A^*$ and thus $A^*$ is a minimizer as claimed.

%Recall that $-C\in \partial \left(\delta\|2A-11^t\|\right)(A^*)$ if and only if it satifies the conditions
%\[ \langle H^* , C\rangle =-2\delta n\text{ and }\|C\|\leq 2\delta \]
%where $H^*:=2A^*-11^t$ and $\|\bullet\|$ is the spectral norm. 
%Both of these conditions are satisfied by setting $C=-\frac{2\delta}{n} H^*$. However this choice of $C$ will not, in general, satisfy the inequalities $\widetilde{\Gamma}_{ij}\leq \widetilde{-\frac{2\delta}{n} H^*}_{ij}=-\frac{2\delta}{n}11^t_{ij}$ for $i\neq j$. 

%We will adjust our candidate for $\widetilde{C}$ by adding to it a transportation matrix $\widetilde{K}$ that will guarantee that all these inequalities are satisfied when $b(\Gamma,\delta)\leq a(\Gamma, \delta)$. Crucially we will choose $\widetilde{K}$ so that $K$ satisfies $KH^*=H^*K=0$ allowing us to control the spectral norm of $C$.

%Since $b(\Gamma,\delta)\leq a(\Gamma,\delta)$ there exists a way to redistribute the quantity $b(\Gamma,\delta)$ by substracting it from the $ij$ for which $-\frac{\delta}{n}\leq \widetilde{\Gamma}_{ij}$  and adding it into those $st$ for which $\widetilde{\Gamma}_{st}\leq -\frac{\delta}{n}$. More specifically, if $b=\widetilde{\Gamma_{ij}}+\frac{2\delta}{n}>0$ then there exists a set $(i_1,j_1),\dots, (i_t,j_t)$ of non-diagonal entries and nonnegative constants $\gamma_{i_s,j_s}$ summing to one 
%such that $\widetilde{\Gamma}_{i_sj_s}+\frac{2\delta}{n}+ b\gamma_{i_s,j_s}\leq 0$. Define the off-diagonal elements of $\widetilde{C}$ by
%\[\widetilde{C}:=-\frac{2\delta}{n} 11^t + \sum_{i_s,j_s} b\gamma_{i_s,j_s}(-e_{ij}+e_{i_s,j_s})\]
%where $e_{ij}$ is the symmetric matrix with one in positions $i,j$ and $j,i$ and zeroes otherwise. More generally, let $B$ be the set of paris $(i,j)$ with $i<j$ such that $\widetilde{\Gamma_{ij}}+\frac{2\delta}{n}>0$. If $b(\Gamma,\delta)\leq a(\Gamma,\delta)$ then for each $ij\in B$ there exist nonnegative constants $\gamma^{(ij)}_{st}$ such that $\sum_{s < t} \gamma^{(ij)}_{st}=1$ and for which the matrix
%\[\widetilde{C}:=-\frac{2\delta}{n}11^t+\sum_{ij\in B} \sum_{s<t} \left(\widetilde{\Gamma_{ij}}+\frac{2\delta}{n}\right)\gamma_{st}^{(ij)} (-e_{ij}+e_{st})\]  
%satisfies $\widetilde{\Gamma}_{ab}\leq \widetilde{C}_{ab}$ for all $a\neq b$. We will show that if $b(\Gamma,\delta)\leq 2\delta$ then the matrix $C$ with off-diagonal entries given by
%\[C_{ab}=-\frac{2\delta}{n}H^*_{ab} + \sum_{ij\in B} \sum_{s<t} \left(\widetilde{\Gamma_{ij}}+\frac{2\delta}{n}\right)\gamma_{st}^{(ij)} \widetilde{(-e_{ij}+e_{st})_{ab}}\]
%lies in $-\partial \left(\delta\|2A-11^t\|_*\right)(A^*)$ for some choice of diagonal, proving the Theorem. 
%For a pair of indices $i<j$ let $\epsilon_{ij}=1$ if $ij\in I$ and $\epsilon_{ij}=-1$ if $ij\in O$. Define the matrix $t_{ij}$ by $t_{ij}=\epsilon_{ij} e_{ij}-e_{ii}-e_{jj}$ and note that $t_{ij}$ satisfies the following three properties: 
%\begin{enumerate}
%\item The equalities $H^*t_{ij}=0=t_{ij}H^*$ hold. If $ij\in O$ this happens only when there are exactly two clusters and this is the only point in the proof where this assumption is used.
%\item The off-diagonal entries of $\widetilde{t_{ij}}$ are equal to those of $e_{ij}$. In particular the off-diagonal entries of $\widetilde{(-e_{ij}+e_{st})_{ab}}$ are always equal to those of $\widetilde{-t_{ij}+t_{st}}$.
%\item The inequality $\|-t_{ij}+t_{st}\|\leq 2$ holds for all $ij$ and $st$. This is immediate via direct calculation (the inequality is strict only if $|\{i,j\}\cap\{s,t\}|\geq 1$ and in this case it can take values of $\sqrt{3}$ and $0$).  
%\end{enumerate}
%If $C$ denotes the matrix given by
%\[C:= -\frac{2\delta}{n}H^* + \sum_{ij\in B} \sum_{s<t} \left(\widetilde{\Gamma_{ij}}+\frac{2\delta}{n}\right)\gamma_{st}^{(ij)} (-t_{ij}+t_{st})\]
%then the following properties hold:
%\begin{enumerate}
%\item The off-diagonal entries agree with those in our previous expression so $\widetilde{\Gamma}_{ab}\leq \widetilde{C}_{ab}$ for all $a\neq b$ and thus $C$ is in the subdifferential of Lemma~\ref{lem: subdiff}.
%\item Since $H^*t_{ij}=0=t_{ij}H^*$ the equality $\langle H^*, C\rangle =\langle H^*,-\frac{2\delta}{n}H^*\rangle = -2\delta n$ holds and moreover
%\[ \| C\|=\max\left(\left\|-\frac{2\delta}{n}H^*\right\|, \left\|\sum_{ij\in B} \sum_{s<t} \left(\widetilde{\Gamma_{ij}}+\frac{2\delta}{n}\right)\gamma_{st}^{(ij)} (-t_{ij}+t_{st})\right\|\right).\] 
%\end{enumerate}
%The operator norm of the first term in the maximum equals $2\delta$ and that of the second term is bounded by $2b(\Gamma,\delta)$ by the triangle inequality and the definition of $b(\Gamma,\delta)$. We conclude that $\|C\|$ is bounded by $2\delta$ because $b(\Gamma,\delta)\leq \delta$. As a result $-C\in \partial\left(\|2A-11^t\|\right)(A^*)$ proving the Theorem.
%\end{proof}

%\mv{Como extendemos este razonamiento al caso de tres o m\'as clusters? Debe ser posible utilizar otras matrices de transporte para este caso que vivan en el kernel. Algo interesante es que con tres clusters es necesario que los promedios de una matriz aniquilada por $H^*$ dentro de cada uno de los bloques $C_i\times C_j$ deban ser cero.} 

We will now prove the general case when there are more than two clusters. The proof will be similar the proof of the previous theorem. We will start with the candidate matrix $\frac{-2\delta}{n}H^*$ and correct it by adding transport matrices that assure that the corrected matrix belong to the subdifferential of $\Delta(A)$ at $A^*$.
The difficulty in applying the tools of the previous theorem to solve the general case is that the matrices $K$ that transport weight from one cluster to another do not, in general, satisfy the relation $KH^* = H^*K = 0$ when there are more than two clusters. This problem can be solved by splitting the matrix $H^*$ into matrices than only take into account $2$ clusters, and using the previous theorem.
We begin recalling the following simple result:

\begin{claim}
Let $a_i,b_i \geq 0, \ i = 1,\dots,p$ be two non-negative, finite sequences of real numbers such that
\[
\sum_{i=1}^{p}a_i \geq \sum_{i=1}^{p}b_i.
\]
Then there exist a finite sequence of reals $c_i$ such that
\begin{itemize}
\item $\sum_{i=1}^p c_i=0.$
\item $a_i \geq b_i+c_i \  \forall i.$
\end{itemize}
\end{claim}


Now we proceed to do the proof. For  clusters $C_s\neq C_t$ define the matrix $H^{C_sC_t}$ whose entries are given by: 
\[
H^{C_iC_j}_{uv}= 
\begin{cases}
1 \text{ if } u,v \in C_s \text{ or } u,v \in C_t. \\
-1 \text{ if } u \in C_s, v \in C_t \text{ or } u \in C_s, v\in C_t. \\
0 \text{ in any other case.}
\end{cases}
\]
Notice that this matrix has $4$ blocs. Two with only $1$ and two with only $-1$. Moreover, its spectral norm is equal to $|C_s+|C_t|$.

\begin{lemma}\label{lemma: transport2}
Assume there are $l$ clusters. Suppose that $ b(\Gamma,\delta) < min(\delta,a(\Gamma,\delta))$.Then,
$A^*$ is a minimizer of the optimization problem $\min_A\Delta(A)$. 
\end{lemma}

\begin{proof}

First of all, observe that
\[
\frac{-2\delta}{n}H^* = \frac{-2\delta}{n}\frac{1}{l-1}\sum_{1\leq s<t\leq l}H^{C_sC_t}.
\]
For each, $C_s\neq C_t$ construct a transport matrix $\Delta_{st}$ as in the previous theorem, as to assure that $\Delta_{st}H^{C_sC_t} = H^{C_sC_t} \Delta_{st}=0$. This can be done since $H^{C_sC_t}$ takes into account only two clusters. Recall that the total amount of weight to be corrected is $b(\Gamma,\delta)$. Let $w_{s,t}$ the weight to be distributed from cluster $s$ to cluster $t$. In the notation of the previous theorem, $w_{st}$ is just the sum of the 
$\gamma_{l_p}^{ij}$ where $l_p\in C_s $ and $ij \in C_t $.


Let 

\[
C:= -\frac{2\delta}{n}H^* + \sum_{i<j}\Delta_{ij} = \frac{-2\delta}{n}\frac{1}{l-1}\sum_{1\leq i<j\leq l}H^{C_iC_j} + \sum_{i<j}\Delta_{ij}
\]

Finally, assume that for each $i<j$, $\frac{1}{l-1}||H^{C_iC_j}||\geq ||\Delta_{ij}||$.

Then,
\[
\begin{aligned}
\left\|C\right \| & = \left\| \frac{-2\delta}{n}\frac{1}{l-1}\sum_{1\leq i<j\leq l}H^{C_iC_j} + \sum_{i<j}\Delta_{ij}  \right \|  \\ & \leq \frac{2\delta}{n(l-1)}\sum_{1\leq i<j\leq l}\left \|H^{C_iC_j}+ \frac{n(l-1)}{2\delta} \Delta_{ij}\right \|  \\
& = \frac{2\delta}{n(l-1)}\sum_{1\leq i<j\leq l} \max (\left\|H^{C_iC_j}\right\|,\left\|\frac{n(l-1)}{2\delta}\Delta_{ij}\right\|)& \\
\end{aligned}
\]
Now notice that 
\[
\begin{aligned}
\delta \geq b(\Gamma,\delta) \text{ therefore } (l-1)n & \geq \frac{(l-1)n}{\delta} b(\Gamma,\delta) \\\text{ which implies that }  \sum_{i<j}\left \|H^{C_iC_j}\right \|  & \geq  \frac{(l-1)n}{\delta}\sum_{i<j}w_{i,j}   \\
 & \geq \frac{(l-1)n}{2\delta}\sum_{i<j}\left \|\Delta_{i,j} \right \|  .
\end{aligned}
\]
By the claim, we can assume without loss of generality that
for each $i<j$,
\[
\left \|H^{C_i,C_j}\right \| \geq  \frac{(l-1)n}{\delta}\left \|\Delta_{i,j} \right \|.
\]

This implies that the last sum reduces to
\[
\frac{2\delta}{n(l-1)}\sum_{1\leq i<j\leq l} \left\|H^{C_iC_j}\right\| = \sum_{1\leq i<j\leq l} \frac{2\delta(|C_i|+|C_j|)}{n(l-1)} = \frac{2\delta(l-1)}{n(l-1)}\sum_{1\leq i<j\leq l}|C_i|+|C_j|=2\delta.
\]
And so
$\|C\| \leq 2\delta$.

\end{proof}
\ddr{toca revisar que $<h^*,C>$ es igual a $-2\delta n$ o eso es obvio?}


\subsection*{Uniqueness of the minimizer}
In this brief section we discuss an important corollary: $A^*$ is the unique minimizer of the optimization problem $min_A\Delta (A)$.
We begin proving a well known lemma. 

\begin{lemma}{\label{lemUniqueness}}
Let f be a convex function defined over a region $D$. Let $\hat{x}$ be a point in it's domain such that the subdifferential of $f$ at $\hat{x}$ is full dimensional and $0$ belongs to it's interior. Then, $\hat{x}$ is the unique minimizer of $f$.
\end{lemma}
\begin{proof}
Let $B_\epsilon(0)$ be a ball of radius $\epsilon$ centered in $0$ and contained in 
$\partial(f)(\hat{x})$. Let $x \in D \text{ with } x \neq \hat{x}$. Let $Q \in \partial(f)(\hat{x})$. By the property of the elements of the subdifferential at a point, 
\[
f(x) \geq f(\hat{x}) + \left \langle Q,x-\hat{x} \right \rangle.
\]
This property holds for every $Q \in \partial(f)(\hat{x})$, so taking supremum we obtain that
\[
f(x) \geq \sup_{Q \in \partial(f)(\hat{x})} f(\hat{x}) + \left \langle Q,x-\hat{x} \right \rangle..
\]
Since $B_\epsilon(0) \subseteq \partial(f)(\hat{x})$ we obtain that

\[
 f(x) \geq f(\hat{x})+\sup_{Q \in B_\epsilon(0)} \left \langle Q,x-\hat{x} \right \rangle = f(\hat{x})+ \epsilon\| x-\hat{x} \|.
\]
As $x \neq \hat{x}$, $\epsilon \| x-\hat{x} \|> 0$. Therefore, $f(x)>f(\hat{x})$ for all $x\in D$ different of $\hat{x}$.

\end{proof}

\begin{remark}{\label{remUnicity}}
Under the conditions of theorem \ref{thm: transport}, The matrix $C^1$ we constructed satisfies the inequalities given by \ref{lem: subdiff} strictly, so $C^1$ is an interior point of $\partial(f)(A^*)$. It follows that the matrix $C$ constructed in \ref{lemma: transport2} also satisfies these inequalities strictly and thefore it is also an interior point of $\partial(f)(A^*)$.
\end{remark}




\begin{cor}
Under the conditions of theorem \ref{thm: transport}, $A^*$ is the unique minimizer of the optimization problem $min_A\Delta (A)$.
\end{cor}
\begin{proof}
By \ref{remUnicity}, the matrix $C$ constructed in \ref{lemma: transport2} is an interior point of  $\partial(f)(A^*)$. Therefore, as the subdifferential of $\Delta$ at $A^*$ is equal to the Minkowski sum of the subdifferentials of $g$ and $f$ at $A^*$,$0=C-C$ is an interior point of $\partial(\Delta)(A^*)$. It follows by lemma \ref{lemUniqueness} that $A^*$ is the unique minimizer of the optimization problem  $min_A\Delta (A)$.


\end{proof}



Using  Theorem \ref{thm: transport} we now estimate the probabilities of perfect recovery of the correct cluster structure. 



\subsection{A bound for recovery probabilities}

In this section we will bound the probability that the correct $A^*$ is not an optimal solution of our proposed optimization problem. A key tool will be the following version of Hoeffding's inequality: If $X_1,\dots, X_T$ are independent random variables with values in $[c_i,d_i]$ and $\Lambda_T:=\sum_{i=1}^T X_i$ then the following inequality holds for all $t\geq 0$ 
\[\PP\{\Lambda_T-\EE[\Lambda_T]\geq t\}\leq \exp\left(-\frac{2t^2}{\sum_{i=1}^T (d_i-c_i)^2}\right).\]




\begin{theorem} Suppose $B_1,\dots, B_N$ are independent and have the same distribution as $\grG$. If $\alpha = \min (|p_t-\frac{1}{2}|, |q-\frac{1}{2}|)$ and $\delta^*$ is the maximum of $a(\delta):=\frac{\delta\left(\alpha-\frac{\delta}{n}\right)^2}{\left(1+\frac{2\delta}{n}\right)}$ in $[0,\alpha n]$ then the probability that $A^*$ is not a minimizer of~(\ref{eqn: WRGS2}) is bounded above by
\[ \exp\left(-\frac{2N(n-1)^3(\alpha n-\delta^*)^2}{n}\right) + e^{-N\left(\delta^* a(\delta^*)\right)}\prod_{i\neq j}\left(1+ (e^{-4a}p_{ij}+q_{ij})^{\frac{N}{2}}\right).\] 
Moreover this quantity decreases exponentially with the sample size $N$.
\end{theorem}
\begin{proof} By Lemma~\ref{lemma: transport2} and the union bound the probability that $A^*$ is not an optimal solution of problem~(\ref{ProbRobusto}) is bounded above by

\begin{equation}\label{Eq: 2Probs}
\PP\{b(\Gamma,\delta)\geq a(\Gamma,\delta)\}+\PP\{b(\Gamma,\delta)\geq \delta\}.
\end{equation}

and we will find upper bounds for the individual terms in~(\ref{Eq: 2Probs}). 
For $t=1,\dots, N$ and $i,j\in [n]$ define 
\[Z^{(t)}_{ij}:=\begin{cases}
-1\text{, if $(B_t)_{ij}=1$}\\
+1\text{, if $(B_t)_{ij}=0$}
\end{cases}\]
and note that for every $i\neq j$ the equality $\sum_{t=1}^N\frac{Z_{ij}^{(t)}}{N} = \Gamma_{ij}$ holds. As a result
\[b(\Gamma,\delta)-a(\Gamma,\delta) = \sum_{i\neq j} \left(\widetilde{\Gamma_{ij}}+\frac{2\delta}{n}\right)= \frac{2\delta n(n-1)}{n} + \sum_{t=1}^N \sum_{i\neq j} \frac{\widetilde{Z^{(t)}_{ij}}}{N}\]
and therefore if $M:=\EE\left[\sum_{t=1}^N \sum_{i\neq j} \frac{\widetilde{Z^{(t)}_{ij}}}{N}\right]$ then the number $M$ is negative and is given by the formula
\[M=(2q-1)2|O|+\sum_{i=1}^l 2\binom{c_i}{2} (1-2p_i)\leq -2\alpha n(n-1)\]  
We can therefore bound the probability in the first term with
 
\[\PP\left\{\frac{2\delta n(n-1)}{n} + \sum_{t=1}^N \sum_{i\neq j} \frac{\widetilde{Z^{(t)}_{ij}}}{N}\geq 0\right\} = \PP\left\{\sum_{t=1}^N \sum_{i\neq j} \frac{\widetilde{Z^{(t)}_{ij}}}{N} - M \geq -2\delta (n-1) - M \right\}\leq\]
\[\leq \exp\left(-2\frac{(-M-2\delta(n-1))^2}{Nn(n-1)(\frac{2}{N})^2}\right)=\exp\left( -\frac{N}{2}\left(1-\frac{1}{n}\right) \left(\frac{-M}{n-1}-2\delta\right)^2\right) \]
where the inequality follows from Hoeffding's inequality applied to the $Nn(n-1)$ independent random variables $\frac{\widetilde{Z_{ij}^{(t)}}}{N}$ which have values in $\left[-\frac{1}{N},\frac{1}{N}\right]$. The inequality applies whenever $-\frac{M}{2(n-1)}>\delta>0$. In particular whenever $0<\delta<\alpha n$ we have
\[\PP\{b(\Gamma,\delta)-a(\Gamma,\delta)\}\leq \exp\left(-\frac{2N(n-1)^3(\alpha n-\delta)^2}{n}\right).\]
Bounding the second term is more involved. Recall that
\[b(\Gamma,\delta)=\sum_{i\neq j} \max\left(\widetilde{\Gamma_{ij}}+\frac{2\delta}{n},0\right).\]
Let $Y_{ij}:=\widetilde{\Gamma_{ij}}+\frac{2\delta}{n}$ and let $X_{ij}:=\max\left(Y_{ij},0\right)$. In order to prove a concentration inequality for the variables $X_{ij}$ we begin by studying their moment generating functions $m_{X_{ij}}(t)$. Note that for every real number $t$ the equality
\[\exp(tX_{ij})= 1_{\{Y_{ij}\leq 0\}} + 1_{\{Y_{ij}\geq 0\}} \exp tY_{ij}\]
holds. Now $Y_{ij}=1+\frac{2\delta}{n}-2\frac{n_{ij}}{N}$ where $n_{ij}$ is a binomial random variable with parameters $N$ and $p_{ij}$ given by
\[
p_{ij}:=\begin{cases}
p_t\text{, if $\{i,j\}\subseteq C_t$}\\
1-q\text{, else}
\end{cases}
\]


As a result taking expected values on both sides of the expression above we conclude that
\[ m_{X_{ij}}(t) \leq \PP\{Y_{ij}\leq 0\} + e^{t\left(1+\frac{2\delta}{n}\right)}\EE\left(e^{-\frac{2t}{N}n_{ij}}1_{\{Y_{ij}\geq 0\}}\right).\]

Using the Cauchy-Schwartz inequality and the known formula for the moment generating function of a binomial random variable it follows that
\[ m_{X_{ij}}(t)\leq \PP\{Y_{ij}\leq 0\}+ e^{t\left(1+\frac{2\delta}{n}\right)}\EE\left(e^{-\frac{4t}{N}n_{ij}}\right)^{\frac{1}{2}}\PP\{Y_{ij}\geq 0\}^{\frac{1}{2}}=\]
\[=\PP\{Y_{ij}\leq 0\}+ e^{t\left(1+\frac{2\delta}{n}\right)}\left(e^{-\frac{4t}{N}}p_{ij} + q_{ij}\right)^{\frac{N}{2}}\PP\{Y_{ij}\geq 0\}^{\frac{1}{2}}\]
where $q_{ij}:=1-p_{ij}$.
By Hoeffding's inequality on Bernoulli random variables we know that
\[\PP\{Y_{ij}\geq 0\}\leq \exp\left(-\frac{N}{2}\left(-\frac{2\delta}{n}-(1-2p_{ij})\right)^2 \right)\leq \exp\left(-2N(\alpha-\delta/n)^2\right)\]
so if $t=aN$ the inequality 
\[e^{t\left(1+\frac{2\delta}{n}\right)}\exp\left(-N(\alpha-\delta/n)^2\right)\leq 1\] 
holds whenever $a\leq \frac{(\alpha-\delta/n)^2}{\left(1+\frac{2\delta}{n}\right)}$ and for all such $a$ we have
\[m_{X_{ij}}(aN)\leq \left(1+ (e^{-4a}p_{ij}+q_{ij})^{\frac{N}{2}}\right)\]
We define $a(\delta):=\frac{(\alpha-\delta/n)^2}{\left(1+\frac{2\delta}{n}\right)}$ and will use it to prove a moment concentration inequality for $b(\Gamma,\delta)$ which will give us a bound on the second term in~(\ref{Eq: 2Probs}). For every $t>0$ we have

\[\PP\left\{b(\Gamma,\delta)\geq \delta\right\}=\PP\left\{\exp\left(t\sum_{i\neq j} X_{ij}\right)\geq e^{t\delta}\right\}\leq e^{-t\delta}\prod_{i\neq j} \EE[e^{tX_{ij}}]= e^{-t\delta}m_{X_{ij}}(t)\]
Choosing $t=a(\delta)N$ and using the previous inequality we see that
\[\PP\left\{b(\Gamma,\delta)\geq \delta\right\}\leq e^{-N\delta a(\delta)}\prod_{i\neq j}\left(1+ (e^{-4a}p_{ij}+q_{ij})^{\frac{N}{2}}\right)\]
Which decreases exponentially in $N$ for any $0\leq \delta\leq n\alpha$. The rate of decrease of the first term is controlled by the positive factor 
\[\delta a(\delta)= \frac{\delta\left(\alpha-\frac{\delta}{n}\right)^2}{\left(1+\frac{2\delta}{n}\right)}.\]
and we let $\delta^*$ be a maximizer of this function.
\end{proof}











\section{An algorithm for Wasserstein nuclear norm summarization }


Solving the optimization problems appearing in Theorem~\ref{Thm: tractable} require solving large semidefinite programs which are beyond the  capacity of standard off-the-shelf software even for relatively small graphs (of say $70$ vertices with $N=4$). One possible reason is that off-the-shelf solvers often use interior point methods, which are highly accurate but often do not scale well. A better alternative, especially well suited for solving is to use first order numerical optimization methods such as ADMM.

In this section, we provide an algorithm specially adapted for the Wassesrtein nuclear norm summarization. It is based on a variation of the \textbf{Alternating Direction Method of Multipliers} (ADMM) called \textbf{Global Variable consensus with regularization}. Our algorithm is derived using the theory given in \cite{boyd2011distributed}.


\subsection{Global Variable consensus with regularization}


The general form of Global Variable consensus with regularization is given by 

\begin{equation}\label{GVC}
\begin{aligned}
& \text{Minimize   } \sum_{i=1}^n f_i(x_i)+g(z) \\
&\text{Subject to: } x_i-z = 0, \ \ i=1,..,N. 
\end{aligned}
\end{equation}



To efficiently solve this optimization problem, ADMM does iterative rounds of minimization of the primal and dual variables. For certain functions, (as is the case for the $l_1$ norm and crucially the nuclear norm) these minimizations problems have actual analytical minimizers. 


The general algorithm to solve this optimization problem is given in \cite{boyd2011distributed} and consists of the following steps:

\begin{enumerate}
\item initialize $x^0_i$, $y^0_i$ for $i=1,..,N$, $\rho>0$ and $z^0$.
\item Set $ \text{Set } \bar{x}^{k} = \frac{1}{N}\sum_{i=1}^N x_i $ and $\bar{y}^{k} = \frac{1}{N}\sum_{i=1}^N y_i $.
\item $x_i^{k+1} = \argmin_{x_i} \left( f_i(x_i) + \left \langle y_i^k,x_i-z_k \right \rangle) +\frac{\rho}{2}\|x_i-z^k\|_2^2\right ).$
\item $z^{k+1} = \argmin_z \left (g(z)+\frac{N\rho}{2}\|z^k-\bar{x}^{k+1} -\frac{1}{\rho}\bar{y}^k \|_2^2 \right ).$
\item $y_i^{k+1} = y_i^k + \rho(x_i^{k+1}-z^{k+1}).$
\end{enumerate}

This algorithm converges under very general circumstances, wich can be easily checked for problem \ref{WRNNS} :
\ddr{las condiciones son que f y g sean propias, cerradas y convexas. mas aun, el lagrangiano no aumentado $L_0$ debe tener un punto de silla. deberiamos incluir una demostracion de esto? mas explicitamente, la condicion es que 
existen $(x^*,y^*,z^*)$ no necesariamente unicos tales que $L_0(x^*,z^*,y)\leq L_0(x^*,z^*,y^*) \leq L_0(x,z,y^*).$ donde 
$L_0(x,y,z)= f(x)+g(z)+ \langle y, Ax-Bz-C\rangle $ en el problema de optmizacion general $min f(x)+g(z)$ sujeto a $Ax+Bz = c$.
}



\subsection{Adaptation to the Wasserstein nuclear norm summarization}


\begin{definition}
Let $f$ be a convex function, $\rho > 0$. The proximal operator of $f$ at $w$ is defined as:

\[
prox_{f,\rho}(w) = \argmin_{z} f(z)+ \frac{\rho}{2}\|z-w\|_2^2
\]

\end{definition}

\begin{claim}
if $\|\cdot\|_*$ denotes the nuclear norm, and if $w$ is a matrix with singular value decomposition $USV^t$ then,
\[
UP_{\epsilon}V^t = \argmin_x \epsilon\|x\|_* + \frac{1}{2}\|x-w\|_2^2.
\]
Where
\[
P_\epsilon=\begin{cases}
x - \epsilon \text{ if } x>\epsilon \\
x+\epsilon \text{ if }x< -\epsilon\\
0 \text{ in any other case. }
\end{cases}
\]
and $P_\epsilon$ is applied component-wise.$P_\epsilon$ is usually called the soft-thresholding operator.
\end{claim}


\begin{claim}
if $\|\cdot\|_1$ denotes the $l_1$ norm, then
\[
prox_{\|\cdot\|_1,\rho}(w) = \argmin_x \|x\|_1 + \frac{\rho}{2}\|x-w\|_2^2 = P_{\frac{1}{\rho}}(w).
\]
Where $P_{\frac{1}{\rho}}$ is applied component-wise. Moreover, if $c$ is any $n \times n$ matrix, then
\[
P_{\frac{1}{\rho}}(w)+c = \argmin_x \|x-c\|_1+\frac{\rho}{2}\|x-w\|_2^2.
\]

\end{claim}

Let $B_1,..,B_N$ of an $i.i.d$ sample of the random graph $\grG$. Let $n$ be the number of vertices of $\grG$ and $1$ denote the vector [1,..,1] of length $n$. To use the formulation given in \ref{GVC} to solve the problem we need the following change of variables:


\begin{itemize}
\item $C_i = 2B_i-11^t$.
\item $Z = 2A-11^t$.
\item $f_i(x_i) = \frac{1}{2}\|x_i-C_i\|_1$.
\item $g(z)=\|z\|_*$.
\end{itemize}

Our optimization problem reduces to the one given in \ref{GVC} and an algorithm to solve it is given by:

\begin{enumerate}
\item $\text{initialize } \rho >0 x_i^0=0, y_i^0=0, \text{for } i=1,..,N \text{ and } z_0= \frac{1}{N}\sum_{i=1}^N C_i$.
\item $ \text{Set } \bar{x}^{k} = \frac{1}{N}\sum_{i=1}^N x_i $ and $\bar{y}^{k} = \frac{1}{N}\sum_{i=1}^N y_i $.
\item $x_i^{k+1}=P_{\frac{1}{2\rho}}(z^k-\frac{1}{\rho}y_i^k-C_i)+C_i$
\item $z^{k+1} = UP_{\frac{\lambda}{N\rho}}(S)V^t$ with $USV^t = SVD(\bar{x}^{k+1}+\frac{1}{\rho}\bar{y}^k)$.
\item $y^{k+1} =y_i^k + \rho(x_i^{k+1}-z^{k+1}). $ 

\end{enumerate}


Using this algorithm, we tested the recovery of graphs generated by the stochastic block model.The simulations are in sink with our theorical results.  Results are shown in figures ... .



\section{Some numerical examples using the stochastic block model.}
\label{Numerics}
\mv{Aca creo que deberia haber ejemplos de tres tipos:
\begin{enumerate}
\item De grafos chiquitos, digamos 20 vertices en los que se vea como cambia el \'optimo en la medida en que el par\'ametro $\delta$ va cambiando.
\item En grafos medianos sus gr\'aficas comparativas de performance para diferentes m\'etricas con pocos samples
\item En grafos grandes algo de cross-validation? Para ver como escoger el $\delta$? Me imagino que si uno va viendo que tan lejos esta de ser entera la solucion entonces uno puede encontrar un rango para $\delta$ donde da resultados ch\'everes.
\end{enumerate}

}

\section{Preliminaries}





\subsection{ Preliminaries on graphs and norms}

By a graph $G$ we mean a finite loopless undirected graph. We say that $G$ is weighted if it is endowed with a function $w: E(G)\rightarrow \RR$ which assigns to every edge a real number in $[0,1]$. If $G$ has $n$ vertices then it is completely specified by its adjacency matrix $A\in \{0,1\}^{n\times n}$ defined by $A_{ij}=1$ if and only if vertices $i,j$ are connected. If $G$ is weighted then we use the term adjacency matrix of $G$ to denote the matrix with entries $A_{i,j}=w(i,j)$. 

If $A$ is a matrix then we use $\|\bullet\|$, $\|\bullet\|_1$ to denote its operator norm and $\ell^1$-norm respectively.





\bibliographystyle{abbrv}
\bibliography{lit}

\end{document}


Our following Lemma gives us a decomposition Theorem for symmetric matrices. This decomposition will be very useful when computing spectral norms in the two-block case.
Suppose that we have only two blocks $C_1$ and $C_2$ and let $u$ be the vector given by
\[u_i=\begin{cases}
1\text{ if $i\in C_1$}
-1\text{ if $i\in C_2$}
\end{cases}
\]
Define $H^*:=uu^t$ and for $i=1,\dots, n$ let $e_{ii}$ be the $n\times n$ matrix whose only nonzero entry is in position $ii$ and has value $1$ 


\begin{lemma} Assume $|C_1|\neq |C_2|$ If $A$ is a symmetric matrix then there exist unique constants $\lambda_1,\dots, \lambda_{n-1}$ and a matrix $K$ such that $KH^*=H^*K=0$ such that
\[A= \frac{\langle H^*,A\rangle }{n^2} H^* +\sum_{i=1}^{n-1}\lambda_1(e_{i,i}-e_{i+1,i+1})+K.\]
And in particular the following equality holds  
\[ \|A\| =\max(A-K,K).\]
\end{lemma}
\begin{proof} The $n$ vectors in $v^tH^*$ and $v^t(e_{i,i}-e_{i+1,i+1})$ for $i=1,\dots, n-1$ are linearly independent in $\RR^n$ because the sum of the entries of all but the first one vanish. It follows that $v^tA$ can be expressed uniquely as a linear combination of them 


\end{proof}



\bibliographystyle{abbrv}
\bibliography{lit}



\end{document}
