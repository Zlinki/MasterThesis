\documentclass[12pt]{amsart}
\usepackage{amstext,amsfonts,amssymb,amscd,amsbsy,amsmath,verbatim}
\usepackage{ifthen}
\usepackage{color,tikz}

\usepackage{amsthm}
\usepackage{latexsym}
\usepackage[all]{xy}
\usepackage{enumerate}
\usepackage{url}

\newtheorem{lemma}{Lemma}[section]
\newtheorem{theorem}[lemma]{Theorem}
\newtheorem{propo}[lemma]{Proposition}

\newtheorem{prop}[lemma]{Proposition}
\newtheorem{cor}[lemma]{Corollary}
\newtheorem{conj}[lemma]{Conjecture}
\newtheorem{claim}[lemma]{Claim}
\newtheorem{claim*}{Claim}
\newtheorem{thm}[lemma]{Theorem}
\newtheorem{defn}[lemma]{Definition}
\newtheorem{example}[lemma]{Example}
\newtheorem{definition}[lemma]{Definition}
\newtheorem{problem}[lemma]{Problem}


\theoremstyle{remark}
\newtheorem{remark}[lemma]{Remark}

\usepackage{geometry,enumerate}
\geometry{a4paper, top=3.5cm, bottom=3cm, left=3cm, right=3cm}

\parindent = 6pt
\parskip = 4pt

% Commands
\newcommand{\isom}{\cong}
\newcommand{\m}{\mathfrak m}
\newcommand{\lideal}{\langle}
\newcommand{\rideal}{\rangle}
\newcommand{\initial}{\operatorname{in}}
\newcommand{\Hilb}{\operatorname{Hilb}}
\newcommand{\Spec}{\operatorname{Spec}}
\newcommand{\im}{\operatorname{im}}
\newcommand{\NS}{\operatorname{NS}}
\newcommand{\Frac}{\operatorname{Frac}}
\newcommand{\ch}{\operatorname{char}}
\newcommand{\Proj}{\operatorname{Proj}}
\newcommand{\id}{\operatorname{id}}
\newcommand{\Div}{\operatorname{Div}}
\newcommand{\tr}{\operatorname{tr}}
\newcommand{\Tr}{\operatorname{Tr}}
\newcommand{\Supp}{\operatorname{Supp}}
\newcommand{\Gal}{\operatorname{Gal}}
\newcommand{\Pic}{\operatorname{Pic}}
\newcommand{\QQbar}{{\overline{\mathbb Q}}}
\newcommand{\Br}{\operatorname{Br}}
\newcommand{\Bl}{\operatorname{Bl}}
\newcommand{\cF}{\mathcal{F}}
\newcommand{\NN}{\mathbb{N}}
\newcommand{\grad}{\nabla}
\DeclareMathOperator*{\argmin}{arg\,min}

\newcommand{\Cox}{\operatorname{Cox}}
\newcommand{\Tor}{\operatorname{Tor}}
\newcommand{\diam}{\operatorname{diam}}
\newcommand{\Hom}{\operatorname{Hom}} %done
\newcommand{\sheafHom}{\mathcal{H}om}
\newcommand{\Gr}{\operatorname{Gr}}
\newcommand{\Osh}{{\mathcal O}}
\newcommand{\kk}{\kappa}
\newcommand{\rank}{\operatorname{rank}}
\newcommand{\codim}{\operatorname{codim}}
\newcommand{\conv}{\operatorname{conv}}
\newcommand{\D}{{\mathcal D}}
\newcommand{\PP}{\mathbb{P}}
\newcommand{\EE}{\mathbb{E}}

\newcommand{\RR}{\mathbb{R}}
\newcommand{\zz}{\mathbb{Z}}
\newcommand{\Sym}{\operatorname{Sym}} %done
\newcommand{\GL}{{GL}}
\newcommand{\grG}{{\mathcal{G}}}

\newcommand{\Syz}{\operatorname{Syz}}
\newcommand{\defi}[1]{\textsf{#1}} % for defined terms


\newcommand{\Bmod}{\ensuremath{B_
\text{mod}}}
\newcommand{\Bint}{\ensuremath{B_\text{int}}}
\newcommand\commentr[1]{{\color{red} \sf [#1]}}
\newcommand\commentb[1]{{\color{blue} \sf [#1]}}
\newcommand\commentm[1]{{\color{magenta} \sf [#1]}}
\newcommand{\ddr}[1]{{\color{blue} \sf $\clubsuit\clubsuit\clubsuit$ Daniel: [#1]}} 
\newcommand{\mv}[1]{{\color{red} \sf $\clubsuit\clubsuit\clubsuit$ Mauricio: [#1]}}

\begin{document}

\author{Daniel De Roux}
\address{
Departamento de matem\'aticas\\
Universidad de los Andes\\
Carrera $1^{\rm ra}\#18A-12$\\ 
Bogot\'a, Colombia
}
\email{d.de1033@uniandes.edu.co}

\author{Mauricio Velasco}
\address{
Departamento de matem\'aticas\\
Universidad de los Andes\\
Carrera $1^{\rm ra}\#18A-12$\\ 
Bogot\'a, Colombia
}
\email{mvelasco@uniandes.edu.co}

\subjclass[2000]{Primary 15A29 % Inverse problems
Secondary 15B52,52A22} % Random matrices, Random Convex sets and integral geometry
\keywords{Compressed sensing, truncated moment problems, Kostlan-Shub-Smale polynomials}

\begin{abstract} We study the problem of discovering the cluster structure of a random graph $\grG$ from an independent sample of size $N$. We propose a Wasserstein robust formulation of this optimization problem and prove that it can be reformulated as a tractable convex optimization problem. We give theoretical performance guarantees for these problems when the Wasserstein metric is induced by the nuclear norm and $\grG$ is distributed according to the stochastic block model explaining its good practical behavior. Finally we present our Julia implementation of the proposed algorithm and use it to analyze the voting patterns of the colombian senate in \mv{period?}.
\end{abstract} 

\title{Graph clustering and the nuclear Wasserstein metric.}
\maketitle

\section{Introduction}



Let $\grG$ be a random graph with $n$ vertices. By a deterministic summary of $\grG$ we mean a (deterministic) graph $H^*$ which, on average, differs from $\grG$ by as few edges as possible. In this article we study the problem of finding deterministic summaries {\it from an independent sample} of $\grG$ of size $N$. More precisely we will address the following problem:

\begin{problem}\label{Prob} Given adjacency matrices $B_1,\dots, B_N$ of an independent sample with the same distribution as $\grG$ find a symmetric matrix $A^*$ in $\argmin_A \EE_{B\sim \grG}[\|A-B\|_1]$. 
\end{problem}

Special cases of this problem arise in cluster detection and in data summarization, both heavily studied in the literature (\mv{refs?}).

A possible approach to problem~\ref{Prob} is to use the samples to use the empirical measure $\hat{\mu}:=\sum_{i=1}^N \frac{1}{N}\delta_{B_i}$ as an approximation of the distribution of $\grG$ and to find a minimizer $A$ of the resulting empirical risk
\[ \EE_{B\sim \mu}[\|A-B\|_1]=\frac{1}{N}\sum_{i=1}^N \|A-B_i\|_1.\]
When the sample size $N$ is not sufficienly large for $\hat{\mu}$ to be a good approximation for the distribution of $\grG$ this approach leads to overfitting. To mitigate this we propose a robust version of the summarization problem by minimizing the worst-case risk when the distribution of $B$ is allowed to vary in a ball $\mathcal{N}_{\delta}(\hat{\mu})$ of radius $\delta>0$ centered at $\hat{\mu}$ in a suitable metric, leading to the following robust optimization problem:

\begin{problem}\label{ProbRobusto} Given adjacency matrices $B_1,\dots, B_N$ of an independent sample with the same distribution as $\grG$ find a symmetric matrix $\overline{A}$ which minimizes  
\[R(A):=\left(\sup_{\nu\in \mathcal{N}_{\delta}(\hat{\mu})} \EE_{B\sim G}[\|A-B\|_1]\right).\] 
\end{problem}


The seminal work of Kuhn et. al. \mv{ref?} shows that using a Wasserstein metric among probability distributions in the problem above leads to a tractable convex optimization problem. Our first result is that this is also true for the robust graph summarization problem when the Wasserstein metric is induced by a semidefinitely representable metric.

\begin{theorem}\label{thm: finiteConvex} Let $K$ be the set of $n\times n$ symmetric matrices with entries in $[0,1]$. If $\delta>0$ then problem~\ref{ProbRobusto} is equivalent to:
\[\min_{(A,s_1,\dots, s_N,\lambda)\in \mathcal{T}}\left( \lambda\delta +\frac{1}{N}\sum_{i=1}^N s_i + \frac{1}{N}\sum \|A-B_i\|_1\right)\]
where $\mathcal{T}$ is the set of $(A,\vec{s},\lambda)$  satisfying the inequalities $\lambda\geq 0$ and $\eta_{\epsilon, i}(\lambda)\leq s_i$ as $i=1,\dots, N$ and $\epsilon$ ranges over all $\{0,1\}$-symmetric matrices and
\[\eta_{\epsilon, i}(\lambda):= \sup_{Y\in K} \left(\langle \epsilon, A-Y\rangle -\lambda \|B_i-Y\|\right)\]
\end{theorem}

The formulation of~\ref{ProbRobusto} in Theorem~\ref{thm: finiteConvex} is a finite-dimensional convex optimization problem which unfortunately contains  an exponential number of constraints. We therefore introduce:
\begin{enumerate}
\item A more tractable simplification which agrees with the original problem whenever the optimal $A$ occurs at a matrix with entries in $\{0,1\}$.
\item A relaxation which takes the form of a regularization of empirical risk minimization.
\end{enumerate}
More specifically, we prove the following 

\begin{theorem}\label{Thm: tractable}If $A$ is a symmetric $\{0,1\}$-matrix then the following statements hold:
\begin{enumerate}
\item{  The number $\sup_{\nu\in B_{\delta}(\hat{\mu})} \EE[\|A-B\|_1]$ is equal to the optimal value of the problem
\[
\min_{\mathcal{H}} \left(\lambda\delta +\frac{1}{N}\sum_{i=1}^N s_i+\frac{1}{N}\sum_{i=1}^N\|A-B_i\|_1\right)
\]
where $\mathcal{H}$ is the set of $(\lambda, s_1,\dots, s_N,\Lambda, W)\in \RR\times\RR^N\times \RR^{n\times n} \times \RR^{n\times n}$ which satisfy the inequalities:
\[
\begin{array}{l}
\|W\|_{*}\leq \lambda\\
\Lambda \geq 0\\
2A-11^t-W + \Lambda \geq 0\\
\|\Lambda\|_{1}\leq s_i-\langle 2A-11^t-W, B_i\rangle\text{ for $i=1,\dots, N$}
\end{array}
\]}
\item The following inequality holds:
\[\sup_{\nu\in B_{\delta}(\hat{\mu})} \EE[\|A-B\|_1]\leq \frac{1}{N}\sum_{i=1}^N\|A-B_i\|_1+ \delta\|2A-11^t\|_*\]
\item If $\|\cdot\|$ is a semidefinitely representable function then the problems:
\begin{equation}\label{eqn: WRGS}
\min_A \min_{\mathcal{H}} \left(\lambda\delta +\frac{1}{N}\sum_{i=1}^N s_i+\frac{1}{N}\sum_{i=1}^N\|A-B_i\|_1\right)\text{ and }
\end{equation}
\begin{equation}\label{eqn: WRGS2}
\min_A\left(\frac{1}{N}\sum_{i=1}^N\|A-B_i\|_1+ \delta\|2A-11^t\|_*\right)
\end{equation}
are semidefinite programming problems.
\end{enumerate}
\end{theorem}

Solving the optimization problems in part $(3)$ of Theorem~\ref{Thm: tractable} leads to a new algorithm for estimating deterministic graph summaries which we call {\it Wasserstein robust graph summarization}. We applied this algorithm to a variety of graphs $G$ distributed according to the stochastic block model and noted that the regularized problem with the Wasserstein metric induced by the nuclear norm, henceforth {\it Wasserstein robust nuclear norm summarization}, was able to recover the correct cluster structure using only very few samples outperforming all other methods known to us. Our next result is a performance guarantee which shows that exact recovery occurs for suitable $\delta>0$ with overwhelming probability explaining the good practical performance of Wasserstein nuclear norm summarization for cluster detection.

Recall that a random graph $\grG$ has distribution given by the stochastic block model in $n$ vertices if there is a partition of $[n]$ into disjoint subsets $C_1,\dots, C_k$, real numbers $0\leq q<\frac{1}{2}<p_1,\dots, p_k\leq 1$ and edges are added independently with probability $p_{ij}$ of joining vertices $i,j$ by an edge where
\[p_{ij}:=\begin{cases}
p_t\text{, if $\{i,j\}\subseteq C_t$}\\
q\text{, else.}
\end{cases}
\] 
Note that a random graph $\grG$ has a unique deterministic summary given by $A^*$ with entries given by
\[
A^*_{ij}:=\begin{cases}
1\text{, if $\{i,j\}\in C_t$ for some $t$}\\
0\text{, otherwise.}
\end{cases}
\]

\begin{theorem} Suppose $B_1,\dots, B_N$ are independent and have the same distribution as $\grG$. If $\alpha = \min (|p_t-\frac{1}{2}|, |q-\frac{1}{2}|)$ and $\delta^*$ is the maximum of $a(\delta):=\frac{\delta\left(\alpha-\frac{\delta}{n}\right)^2}{\left(1+\frac{2\delta}{n}\right)}$ in $[0,\alpha n]$ then the probability that $A^*$ is not a minimizer of~(\ref{eqn: WRGS2}) is bounded above by
\[ \exp\left(-\frac{2N(n-1)^3(\alpha n-\delta^*)^2}{n}\right) + e^{-N\left(\delta^* a(\delta^*)\right)}\prod_{i\neq j}\left(1+ (e^{-4a}\widetilde{p_{ij}}+\widetilde{q_{ij}})^{\frac{N}{2}}\right).\] 
Moreover this quantity decreases exponentially with the sample size $N$.
\end{theorem}

The key point of the Theorem, discussed at length in Section~\ref{}, is that we were able to understand why the nuclear norm is a good regularizer for graph summarization problems, namely because the subdifferential of the regularizer at $A^*$ is sufficiently rich so as to contain enough transportation matrices. 

Next we focus on the practical performance of the proposed algorithm. Problems~(\ref{eqn: WRGS}) and~(\ref{eqn: WRGS2}) require solving large semidefinite programs which are beyond the  capacity of standard off-the-shelf software even for relatively small graphs (of say $40$ vertices with $N=4$). One possible reason is that off-the-shelf solvers often use interior point methods. A better alternative, especially well suited for solving~\ref{WNNC} is to use first order numerical optimization methods such as ADMM. In Section~\ref{Numerics} we adapt the ADMM algorithm for our regularized problem and present our open source Julia implementation allowing us to run the algorithm in graphs of up to 10000 vertices on a common laptop. In Section~\ref{Senadores} we use this algorithm to find a deterministic summary of voting patterns in the colombian senate \mv{Period}. 

\subsection{Acknowledgments.}
We wish to thank Fabrice Gamboa, Mauricio Junca and Thierry Klein for useful conversations during the completion of this project. \mv{CienciaPolitica}.
D. De Roux was partially supported by a grant from Facultad de Ciencias, Universidad de los Andes. M. Velasco was partially supported by research funds from Universidad de los Andes. 

\section{Preliminaries}


\subsection{ Preliminaries on graphs}

By a graph $G$ we mean a finite loopless undirected graph. We say that $G$ is weighted if it is endowed with a function $w: E(G)\rightarrow \RR$ which assigns to every edge a real number in $[0,1]$. If $G$ has $n$ vertices then it is completely specified by its adjacency matrix $A\in \{0,1\}^{n\times n}$ defined by $A_{ij}=1$ if and only if vertices $i,j$ are connected. If $G$ is weighted then we use the term adjacency matrix of $G$ to denote the matrix with entries $A_{i,j}=w(i,j)$. 

If $A$ is a matrix then we use $\|\bullet\|$, $\|\bullet\|_1$ to denote its operator norm and $\ell^1$-norm respectively.




\section{A description of the problem}

By a random graph $B$ we mean a random variable $B$ taking values on the set of adjacency matrices of graphs (i.e. symmeric matrices in $\{0,1\}^{n\times n}$ with zero diagonal). By a random weighted graph we mean a random variable taking values in the adjacency matrices of weighted graphs (i.e. symmetric matrices with $0$ diagonal all of whose off-diagonal entries lie in $[0,1]$). 

\begin{definition} Let $B$ be a random weighted graph and let $A$ be a (weighted) adjacency matrix. Define the risk of choosing $A\in \{0,1\}^n$ as a deterministic summary of $B$ as
\[R(A):=\EE[\|A-B\|_1].\]
We say that $A^*$ is an optimal summary of a random graph $B$ in the set $S$ if it is a minimizer of the optimization problem $\min_{A\in S} R(A)$.
\end{definition}


\section{Recovering cluster structures in the stochastic block model}

Suppose $B$ is a random (undirected loopless) graph on $n$ vertices generated by the stochastic block model. This means that we fix a set partition $C_1,\dots, C_l$ of $[n]$ into sets we call clusters and real numbers $0\leq p_i,\bar{p}\leq 1$ for $i=1,\dots, l$. The edges of $B$ are independent random variables and an edge joins vertices $i,j$ with probability $p_t$ if $\{i,j\}\subseteq C_t$ for some cluster $C_t$ and with probability $\bar{p}$ if $\{i,j\}$ is not contained in any $C_t$. We let $O\subseteq [n]\times [n]$ be the set of pairs of vertices which are not simultaneously contained in any cluster. 

For an integer $N$ let $B_1,\dots, B_N$ be an independent sample of $N$ graphs with the distribution of $B$. Let $A^*$ be the $n\times n$ matrix with entries in $\{0,1\}$ which captures the underlying cluster structure, namely $A^*_{ij}=1$ iff there is a cluster $C_t$ which contains both $ij$.
\ddr{or if i=j. This is important as proofs are clearer if we consider that the entries in the diagonal are not in I or in O.}
In this section we study the probability, as a function of $\delta$ that the optimization problem $\min_A\Delta(A)$
\[\Delta(A)= \delta\|2A-11^t\|_{*}+\frac{1}{N}\sum_{k=1}^N\|A-B_k\|_1\] 
has the correct cluster structure $A^*$ as a minimizer. For vertices $i,j\in [n]$ define $n_1(ij)$ (resp. $n_0(ij)$) the random variables which count the number of times that a given pair is (resp is not) an edge of some $B_j$, $j=1,\dots N$. Note that the $n_q(ij)$ for $q=0,1$ are binomial random variables.  

\begin{lemma} The following statements hold:
\label{lem: subdiff}
\begin{enumerate}
\item The subdifferential of $\frac{1}{N}\sum_{k=1}^N\|A-B_k\|_1$ at $A^*$ is the set of symmetric matrices $C$ satisfying the inequalities
\[ \frac{n_0(ij)-n_1(ij)}{N}\leq C_{ij}\leq 1 \text{, if $\{i,j\}\subseteq C_t$ for some $t$,}\]
\[-1\leq C_{ij} \leq \frac{n_0(ij)-n_1(ij)}{N} \text{ if $\{i,j\}$ does not belong to any cluster and } \]
\[-N \leq C_{ii} \leq N \text{ for all }i.\]

\item The subdifferential of $\delta\|2A-11^t\|_{*}$ at $A^*$ is given by the set of symmetric matrices of the form $2\delta C$ where $C$ has spectral norm $\|C\|\leq 1$ and satisfies $\langle C, 2A-11^t\rangle = n$.

\end{enumerate}

\end{lemma}
\begin{proof} $(1)$ Since the subdifferential is additive it suffices to understand the subdifferential of the absolute value. If $i,j\in C_t$ then $A^*_{ij}=1$ and the entry $ij$ of the subdifferential of the sum at $A^*$ is $[-1,1]$ for each $B_i$ containing the edge and it is $1$ for each $B_i$ for which $(ij)$ is not an edge. \ddr{vale la pena mencionar el caso i=j o es obvio la propiedad en la diagonal?} If $i,j$ is not contained in any cluster then $A^*_{ij}=0$ and the entry $ij$ of the subdifferential of the sum at $A^*$ is $-1$ for each $B_i$ which contains the edge $ij$ and $[-1,1]$ for each $B_i$ which does not, proving the claim. $(2)$ It is easy to prove that the subdifferential of any norm $\|\bullet\|$ at a point $X$ is given by those $C$ for which the dual norm $\|C\|_*\leq 1$ and $\langle C,X\rangle =\|X\|$. Claim $(2)$ follows because $\|2A-11^t\|=Tr(2A-11^t)=n$ where the first equality holds since $2A-11^t$ is positive semidefinite.\mv{Esto es obvio con solo dos clusters (la matriz es $uu^t$ donde $u$ es el vector con $1$'s en un cluster y $-1$'s en el complemento pero hay que demostrarlo para tres o mas)}.
\end{proof}

%\begin{lemma} If $\delta=0$ then 
%\[\PP\{A^*\in\argmin\Delta \} =\prod_{ij\in O} \PP\{n_1(ij)\leq n_0(ij)\} \prod_{t=1}^k \left(\prod_{(ij)\in C_t}\PP\{n_0(ij)\leq n_1(ij)\}\right)\]
%where $\PP\{n_0(ij)\leq n_1(ij)\}$ is given by the following formula 
%\mv{Ejercicio para Daniel: Encontrar una f\'ormula, es la probabilidad de que haya mas caras que sellos en $N$ lanzamientos de una moneda trucada donde las probabilidades de la moneda dependen s\'olo de la arista $ij$}.
%\end{lemma}

%\begin{proof} The matrix $A^*$ is a minimizer of the above convex function if and only if its subdifferential at $A^*$ contains the matrix $0$. By part $(1)$ of the previous Lemma this occurs if and only if $\frac{n_0(ij)-n_1(ij)}{N}\leq 0$ for $(ij)$ in a cluster and 
%$\frac{n_0(ij)-n_1(ij)}{N}\geq 0$ for $(ij)$ in $O$. Independence of the edges then implies the above formula.  
%\end{proof}
%\begin{remark} Could the set of minimizers be larger? If $A'$ has at least one entry $A_{ij}\in (0,1)$ the corresponding component in the subgradient is the constant $n_0(ij)-n_1(ij)$ and this equals zero with much smaller probability, precisely when both terms equal to $\frac{N}{2}$. In particular it is impossible if $N$ is odd and in this case $A^*$ is the only minimizer.
%\end{remark}

%Next we ask whether it is possible to increase the probability of correct recovery by allowing $\delta>0$. By Lemma~\ref{lem: subdiff} $A^*\in \argmin \Delta$ if and only if there exists $S$ in the subdifferential of $\delta\|2A-11^t\|_*$ at $A=A^*$ such that $-S$ belongs to the subdifferential of $\frac{1}{N}\sum_{k=1}^N\|A-B_k\|_1$ at $A^*$. 

%The most immediate way to do this would be to find an $S$ with $S_{ij}\leq 0$ negative on edges $ij\in C_t$ and $S_{ij}\geq 0$ on edges of $O$. More precisely we would like to find a best such $C$ by solving the optimization problem:

%\[
%\min \left(\sum_{t}\sum_{(ij)\in C_t} C_{ij}\right)-\sum_{(ij)\in O} C_{ij} \text{ s.t. $\|B\|\leq 1$, $\langle C, 2A^*-11^t\rangle = n$} 
%\]

%However it is easy to see that we cannot do this improvement simultaneously in all components, because $n=\langle S, 2A-11^t\rangle = Tr(S)+\sum_{ij \in O^c} S_{ij} -\sum_{ij \in O} S_{ij}\leq n+u$ where $u$ is the objective function in the problem above. We conclude that $u$ must be nonnegative so we cannot improve simultaneously in all directions at once.
In the following section we discuss how tradeoffs between components explain the improved recovery probability induced by the spectral norm.

\section{Estimating recovery probabilities}

Let $\Gamma$ be the symmetric matrix with zero diagonal and off-diagonal entries given by
$\Gamma_{ij}=\frac{n_0(ij)-n_1(ij)}{N}$. By Lemma~\ref{lem: subdiff} a symmetric matrix $C$ lies in the subdifferential if and only if it satisfies the inequalities
\[ \Gamma_{ij}\leq C_{ij}\leq 1 \text{, if $\{i,j\}\subseteq C_t$ for some $t$,}\]
\[-1\leq C_{ij} \leq \Gamma_{ij} \text{ if $\{i,j\}$ does not belong to any cluster and } \]
\[-N \leq C_{ii} \leq N \text{ for all }i.\]
To simplify these inequalities we define a linear operator $\widetilde{\bullet}$ on symmetric matrices by the formula
\[ \widetilde{A} = 
\begin{cases}
A_{ij}\text{ if $i=j$ or $ij\in I$ and}\\
-A_{ij}\text{ if $ij\in O$.} 
\end{cases}
\]
in this language $C_{ij}$ belongs to the subdifferential if and only if $\widetilde{\Gamma}_{ij}\leq \widetilde{C}_{ij}$ for $i\neq j$.
The following key result gives sufficient conditions for the true cluster structure $A^*$ to be a minimizer of the proposed optimization problem. In order to describe it we introduce the following notation. 

\begin{definition} Let $\delta$ be a positive real number. For a symmetric matrix $\Gamma$ define the quantities
\[b(\Gamma,\delta):=\sum_{i\neq j} \max\left(\widetilde{\Gamma_{ij}}+\frac{2\delta}{n},0\right)
\text{ and } a(\Gamma,\delta):=\sum_{i\neq j} \max\left(-\widetilde{\Gamma_{ij}}-\frac{2\delta}{n},0\right)
\]\end{definition}
The quantity $b(\Gamma,\delta)$ (resp. $a(\Gamma,\delta)$) measures the total amount by which the matrix $\widetilde{-\frac{2\delta}{n}11^t}$ fails (resp. succeeds) to be in the subdifferential of Lemma~\ref{lem: subdiff} in the sense that it sums over all $ij$ the amount by which the inequalities $\widetilde{\Gamma_{ij}}\leq \frac{2\delta}{n} 11^t$ fail (resp. succeed). The key point of the following Theorem is that if the inequality fails by less than it succeeds then the subdifferential of the spectral norm is sufficiently rich so as to allow us to redistribute these quantities. In this sense the following Theorem explains the success of the spectral norm in cluster recovery algorithms. 


Recall that 
\[\Delta(A)= \delta\|2A-11^t\|_{*}+\frac{1}{N}\sum_{k=1}^N\|A-B_k\|_1\] 
And define functions $f$, $g$:
\[
f:= \frac{1}{N}\sum_{k=1}^N\|A-B_k\|_1, \ \ g:= \delta\|2A-11^t\|_{*}
\]
In this vocabulary, we have that
\begin{equation}\label{SumaSubDif}
\partial(\Delta) = \partial(f)(A^*)+\partial(g)(A^*)
\end{equation}




\begin{theorem}\label{thm: transport} [Alternative proof of theorem 6.2] Assume there are only two clusters. Let $\delta>0$ with $ \left(\frac{\delta}{n}+b(\Gamma,\delta) \right)  < \frac{N}{2}$. If $b(\Gamma,\delta)< \min(\delta, a(\Gamma,\delta))$ then $A^*$ is a minimizer of the optimization problem $\min_A\Delta(A)$.
\end{theorem} 


\begin{proof}
We will show that there exists a matrix $C $ such that 
$-C \in \partial(g)(A^*) $ for which $\widetilde{\Gamma}_{ij}\leq \widetilde{C_{ij}}$ for $i\neq j$. This implies that $C_{ij} \in \partial(f)(A^*)$ and therefore $0=C-C$ belongs to the subdifferential $\partial(\Delta)(A^*)$ and thus $A^*$ is a minimizer of $\Delta(A)$.

Recall that $-C \in  \partial(g)(A^*) $ if and only if 
\[
\left \langle H^*,C \right \rangle = -2\delta n \text{ and } \|C\| \leq 2\delta
\]
where $H^* = 2A-11^t$ and $\|\bullet\|$ is the spectral norm.

Notice that both these conditions are satisfied by setting
$C^0 = -\frac{2\delta}{n}H^*$ as
\[
 \frac{-2\delta}{n}\left \langle H^*,H^* \right \rangle =  \frac{-2\delta}{n}n^2 = -2\delta n \text{ and } \|C^0\|= \frac{2\delta}{n}\|H^*\|= 2\delta.
\]

However this choice of $C^0$ will not, in general, satisfy the inequalities $\widetilde{\Gamma}_{ij}\leq \widetilde{-\frac{2\delta}{n} H^*}_{ij}$ for $i\neq j$. 
Therefore, we will correct our candidate matrix $C^0$ so that it satisfies these inequalities and still belongs to $\partial(g)(A^*)$.

To do this, we will construct a matrix $K$ and add it to $C^0$. Crucially, $K$ will satisfy that $KH^* = H^*K = 0$ so that we can control the spectral norm of $C^0+K$.

Let $i<j$ and 
define the symmetric matrix  $e^{ij}$ as follows:
\begin{equation}
e^{ij}=\begin{cases}
1 \text{ if }(i,j) \in I. \\
-1 \text{ if }(i,j) \in O. \\
-1 \textit{ in the entry } ii \text{ and in } jj. \\
0 \text{ otherwise}. 
\end{cases}
\end{equation}

Observe that for any $(i,j)$ the matrix $e^{ij}$ satisfies the following  properties:

\begin{enumerate}
\item  $H^*e^{ij}=0=e^{ij}H^*$. 
\item The inequality $\|-e^{ij}+e^{st}\|\leq 2$ holds for all $ij$ and $st$. This is immediate noting that the spectral norm is bounded by the product of the induced $1-norm$ and the induced $\infty-$norm. (The inequality is strict only if $|\{i,j\}\cap\{s,t\}|\geq 1$ and in this case it can take values of $\sqrt{3}$ and $0$).  
\end{enumerate}

The idea is to use that $b(\Gamma,\delta) < a(\Gamma,\delta)$ so there exists a way to redistribute the quantity $b(\Gamma,\delta)$ by subtracting it from the $ij$ for which $\frac{2\delta}{n}< \widetilde{\Gamma}_{ij}$  and adding it into those $st$ for which $\widetilde{\Gamma}_{st}\leq \frac{2\delta}{n}$.


Let $U$ be the set of entries $\{i,j\}$ where $\widetilde{\Gamma}_{ij}+ \frac{2\delta}{n} \leq 0$ and $V$ be the  set of entries $\{i,j\}$ where $-(\widetilde{\Gamma}_{ij}+ \frac{2\delta}{n}) > 0$.
Observe that 

\[
b(\Gamma,\delta) = \sum_{ij \in U}( \widetilde{\Gamma}_{ij}+ \frac{2\delta}{n}) \text{ and } a(\Gamma,\delta) =  \sum_{ij \in V} -(\widetilde{\Gamma}_{ij} + \frac{2\delta}{n}).
\]

 By hypothesis $b(\Gamma,\delta)< a(\Gamma,\delta)$.
Let $l_1,..,l_k$ be an enumeration of $V$ were $k$ is it's cardinality. For each entry $ij \in U$, there exists nonnegative coefficients $\gamma^{ij}_{l_1},..,\gamma^{ij}_{l_k}$ such that:

\[
 \widetilde{\Gamma}_{ij}+\frac{2\delta}{n}-\gamma^{ij}_{l_1}-...-\gamma^{ij}_{l_k} < 0. 
\]
and that such that for each  $\gamma_{l_p}$ with $p \in \{1,..,k\}$
\begin{equation}{\label{desA}}
-(\widetilde{\Gamma}_{l_p}+\frac{2\delta}{n})-\sum_{ij \in U}\gamma^{ij}_{l_p}>0.
\end{equation}




For $ij \in U$ define the matrix
\[
W^{ij}:= \gamma^{ij}_{l_1}(\widetilde{e^{ij}-e^{l_1}} )+...+\gamma^{ij}_{l_k}(\widetilde{e^{ij}-e^{l_k}})
\]
Given that $ij \in U$, $l_1,..,l_k \in V$ and the sets $U$ and $V$ are disjoint, the support of $e^{ij}$ is disjoint from the support of any of the matrices $e^{l_1},..,e^{l_k}$ (except probably at the diagonal). In particular, if $ij\in I$ the entry $ij$ of the matrix $W^{ij}$, namely $W^{ij}_{ij}$ is equal to:
\[
\gamma^{ij}_{l_1}(1-0)+...+\gamma^{ij}_{l_k}(1-0)= \gamma^{ij}_{l_1}+...+\gamma^{ij}_{l_k}.
\]
and if  $ij\in O$,
\[
\gamma^{ij}_{l_1}(-1-0)+...+\gamma^{ij}_{l_k}(-1-0)= -\gamma^{ij}_{l_1}-...-\gamma^{ij}_{l_k}.
\]

Define the matrix $C^1$ as:
\[C^1:= C^0 + \sum_{ij \in U} W^{ij} \]
We will now verify that $C^1 \in \partial(f)(A^*)$.
For $ij \in V$, we have by definition of $V$ and by equation \ref{desA} that $\widetilde{\Gamma}_{ij}< \widetilde{C}^1_{ ij}$.  

Let $ij \in I\cap U$. We have that $\Gamma_{ij}+\frac{2\delta}{n}>0$.
Now, the entry of $C^1$ in $ij$ is given by:
\[
C^1_{ij} = -\frac{2\delta}{n}H_{ij}+W^{ij}_{ij} = - \frac{2\delta}{n}+\gamma^{ij}_{l_1}+...+\gamma^{ij}_{l_k}
\]
By the construction of the $\gamma$'s, we have that
\[
\Gamma_{ij}+\frac{2\delta}{n}-\gamma^{ij}_{l_1}-...-\gamma^{ij}_{l_k}<0
\]
it follows that 
\[
\Gamma_{ij}-C^1_{ij}<0
\]
so that
\[
C^1_{ij}>\Gamma_{ij}.
\]
For $ij \in O\cap U$, we have that $\frac{2\delta}{n}-\Gamma_{ij}>0$.
The entry of $C^1$ in $ij$ is given by:
\[
C^1_{ij}=-\frac{2\delta}{n}H_{ij}+W^{ij}_{ij} = \frac{2\delta}{n}-\gamma^{ij}_{l_1}-...-\gamma^{ij}_{l_k}
\]
Since
\[
\frac{2\delta}{n}-\Gamma_{ij}-\gamma^{ij}_{l_1}-...-\gamma^{ij}_{l_k}<0
\]
it follows that
\[
-C_{ij}>-\Gamma_{ij}.
\]

and that $\widetilde{\Gamma}_{ij} < \widetilde{C}^1_{ ij}$ for $i\neq j$. 

It remains to show that the entries of the diagonal of $C^1$ are bounded by $n$. The diagonal entries $\{ss\}$ of $C^1$ are given by
\[
-\frac{2\delta}{n}+ \sum_{ij\in U} W^{ij}_{ss}.
\]


Notice that by the definition of $W^{ij}$, each of the matrices $(\widetilde{e^{ij}-e^{l_p}} )$ has, in the worst case, a $-2$ in the entry $ss$ so the entry $ss$ of $C^1$ is bounded below by 
\[
-\frac{2\delta}{n}+-2 \left(\sum_{ij\in U}\sum_{l_p \in V} \gamma_{l_p}^{ij}\right)
\]
the quantity in the parentheses is all the weight that we have to distribute, i.e $b(\Gamma,\delta)$.
Therefore, 
\[
C^1 \geq -\frac{2\delta}{n}-2(b(\Gamma,\delta)) =-2\left(\frac{\delta}{n}+b(\Gamma,\delta)\right).
\]
By hypothesis, $ \frac{N}{2} > \left(\frac{\delta}{n}+b(\Gamma,\delta) \right) $ so we obtain that
\[
C^1_{ss} > -N.
\]
It is obvious that $N > C^1_{ss}$.
We conclude that $C^1 \in \partial(f)(A^*)$.

For each $ij$, $H^*e^{ij}=0=e^{ij}H^*$ so the equality $\langle H^*, C^1\rangle =\langle H^*,-\frac{2\delta}{n}H^*\rangle = -2\delta n$ holds and moreover

\[ \| C^1\|=\max\left(\left\|-\frac{2\delta}{n}H^*\right\|, \left\|\sum_{ij\in U} W^{ij} \right\|\right).
\]


The operator norm of the first term in the maximum equals $2\delta$ and that of the second term is bounded by $2b(\Gamma,\delta)$ by the triangle inequality and the definition of $b(\Gamma,\delta)$. We conclude that $\|C\|$ is bounded by $2\delta$ because $b(\Gamma,\delta)\leq \delta$. As a result $-C\in \partial\left(g\right)(A^*)$ proving the Theorem. Note that $C^1$ satisfies all the inequalities that define the membership to $\partial(f)(A^*)$ in \ref{lem: subdiff} strictly, so $C^1$ belongs is interior point of $\partial(f)(A^*)$.





\end{proof}






%\begin{theorem}\label{thm: transport} Assume there are only two clusters. If $b(\Gamma,\delta)\leq \min(\delta, a(\Gamma,\delta))$ then $A^*$ is a minimizer of the optimization problem $\min_A\Delta(A)$. 
%\end{theorem}
%\begin{proof} We will show that there exists a matrix $C_{ij}$ such that $-C_{ij}\in \partial \left(\delta\|2A-11^t\|\right)(A^*)$ for which $\widetilde{\Gamma_{ij}}\leq \widetilde{C_{ij}}$ for $i\neq j$. It will then follow that $0=C-C$ belongs to the subdifferential of $\Delta(A)$ at $A^*$ and thus $A^*$ is a minimizer as claimed.

%Recall that $-C\in \partial \left(\delta\|2A-11^t\|\right)(A^*)$ if and only if it satifies the conditions
%\[ \langle H^* , C\rangle =-2\delta n\text{ and }\|C\|\leq 2\delta \]
%where $H^*:=2A^*-11^t$ and $\|\bullet\|$ is the spectral norm. 
%Both of these conditions are satisfied by setting $C=-\frac{2\delta}{n} H^*$. However this choice of $C$ will not, in general, satisfy the inequalities $\widetilde{\Gamma}_{ij}\leq \widetilde{-\frac{2\delta}{n} H^*}_{ij}=-\frac{2\delta}{n}11^t_{ij}$ for $i\neq j$. 

%We will adjust our candidate for $\widetilde{C}$ by adding to it a transportation matrix $\widetilde{K}$ that will guarantee that all these inequalities are satisfied when $b(\Gamma,\delta)\leq a(\Gamma, \delta)$. Crucially we will choose $\widetilde{K}$ so that $K$ satisfies $KH^*=H^*K=0$ allowing us to control the spectral norm of $C$.

%Since $b(\Gamma,\delta)\leq a(\Gamma,\delta)$ there exists a way to redistribute the quantity $b(\Gamma,\delta)$ by substracting it from the $ij$ for which $-\frac{\delta}{n}\leq \widetilde{\Gamma}_{ij}$  and adding it into those $st$ for which $\widetilde{\Gamma}_{st}\leq -\frac{\delta}{n}$. More specifically, if $b=\widetilde{\Gamma_{ij}}+\frac{2\delta}{n}>0$ then there exists a set $(i_1,j_1),\dots, (i_t,j_t)$ of non-diagonal entries and nonnegative constants $\gamma_{i_s,j_s}$ summing to one 
%such that $\widetilde{\Gamma}_{i_sj_s}+\frac{2\delta}{n}+ b\gamma_{i_s,j_s}\leq 0$. Define the off-diagonal elements of $\widetilde{C}$ by
%\[\widetilde{C}:=-\frac{2\delta}{n} 11^t + \sum_{i_s,j_s} b\gamma_{i_s,j_s}(-e_{ij}+e_{i_s,j_s})\]
%where $e_{ij}$ is the symmetric matrix with one in positions $i,j$ and $j,i$ and zeroes otherwise. More generally, let $B$ be the set of paris $(i,j)$ with $i<j$ such that $\widetilde{\Gamma_{ij}}+\frac{2\delta}{n}>0$. If $b(\Gamma,\delta)\leq a(\Gamma,\delta)$ then for each $ij\in B$ there exist nonnegative constants $\gamma^{(ij)}_{st}$ such that $\sum_{s < t} \gamma^{(ij)}_{st}=1$ and for which the matrix
%\[\widetilde{C}:=-\frac{2\delta}{n}11^t+\sum_{ij\in B} \sum_{s<t} \left(\widetilde{\Gamma_{ij}}+\frac{2\delta}{n}\right)\gamma_{st}^{(ij)} (-e_{ij}+e_{st})\]  
%satisfies $\widetilde{\Gamma}_{ab}\leq \widetilde{C}_{ab}$ for all $a\neq b$. We will show that if $b(\Gamma,\delta)\leq 2\delta$ then the matrix $C$ with off-diagonal entries given by
%\[C_{ab}=-\frac{2\delta}{n}H^*_{ab} + \sum_{ij\in B} \sum_{s<t} \left(\widetilde{\Gamma_{ij}}+\frac{2\delta}{n}\right)\gamma_{st}^{(ij)} \widetilde{(-e_{ij}+e_{st})_{ab}}\]
%lies in $-\partial \left(\delta\|2A-11^t\|_*\right)(A^*)$ for some choice of diagonal, proving the Theorem. 
%For a pair of indices $i<j$ let $\epsilon_{ij}=1$ if $ij\in I$ and $\epsilon_{ij}=-1$ if $ij\in O$. Define the matrix $t_{ij}$ by $t_{ij}=\epsilon_{ij} e_{ij}-e_{ii}-e_{jj}$ and note that $t_{ij}$ satisfies the following three properties: 
%\begin{enumerate}
%\item The equalities $H^*t_{ij}=0=t_{ij}H^*$ hold. If $ij\in O$ this happens only when there are exactly two clusters and this is the only point in the proof where this assumption is used.
%\item The off-diagonal entries of $\widetilde{t_{ij}}$ are equal to those of $e_{ij}$. In particular the off-diagonal entries of $\widetilde{(-e_{ij}+e_{st})_{ab}}$ are always equal to those of $\widetilde{-t_{ij}+t_{st}}$.
%\item The inequality $\|-t_{ij}+t_{st}\|\leq 2$ holds for all $ij$ and $st$. This is immediate via direct calculation (the inequality is strict only if $|\{i,j\}\cap\{s,t\}|\geq 1$ and in this case it can take values of $\sqrt{3}$ and $0$).  
%\end{enumerate}
%If $C$ denotes the matrix given by
%\[C:= -\frac{2\delta}{n}H^* + \sum_{ij\in B} \sum_{s<t} \left(\widetilde{\Gamma_{ij}}+\frac{2\delta}{n}\right)\gamma_{st}^{(ij)} (-t_{ij}+t_{st})\]
%then the following properties hold:
%\begin{enumerate}
%\item The off-diagonal entries agree with those in our previous expression so $\widetilde{\Gamma}_{ab}\leq \widetilde{C}_{ab}$ for all $a\neq b$ and thus $C$ is in the subdifferential of Lemma~\ref{lem: subdiff}.
%\item Since $H^*t_{ij}=0=t_{ij}H^*$ the equality $\langle H^*, C\rangle =\langle H^*,-\frac{2\delta}{n}H^*\rangle = -2\delta n$ holds and moreover
%\[ \| C\|=\max\left(\left\|-\frac{2\delta}{n}H^*\right\|, \left\|\sum_{ij\in B} \sum_{s<t} \left(\widetilde{\Gamma_{ij}}+\frac{2\delta}{n}\right)\gamma_{st}^{(ij)} (-t_{ij}+t_{st})\right\|\right).\] 
%\end{enumerate}
%The operator norm of the first term in the maximum equals $2\delta$ and that of the second term is bounded by $2b(\Gamma,\delta)$ by the triangle inequality and the definition of $b(\Gamma,\delta)$. We conclude that $\|C\|$ is bounded by $2\delta$ because $b(\Gamma,\delta)\leq \delta$. As a result $-C\in \partial\left(\|2A-11^t\|\right)(A^*)$ proving the Theorem.
%\end{proof}

%\mv{Como extendemos este razonamiento al caso de tres o m\'as clusters? Debe ser posible utilizar otras matrices de transporte para este caso que vivan en el kernel. Algo interesante es que con tres clusters es necesario que los promedios de una matriz aniquilada por $H^*$ dentro de cada uno de los bloques $C_i\times C_j$ deban ser cero.} 

We will now prove the general case when there are more than two clusters. The proof will be similar the proof of the previous theorem. We will start with the candidate matrix $\frac{-2\delta}{n}H^*$ and correct it by adding transport matrices that assure that the corrected matrix belong to the subdifferential of $\Delta(A)$ at $A^*$.
The difficulty in applying the tools of the previous theorem to solve the general case is that the matrices $K$ that transport weight from one cluster to another do not, in general, satisfy the relation $KH^* = H^*K = 0$ when there are more than two clusters. This problem can be solved by splitting the matrix $H^*$ into matrices than only take into account $2$ clusters, and using the previous theorem.
We begin recalling the following simple result:

\begin{claim}
Let $a_i,b_i \geq 0, \ i = 1,\dots,p$ be two non-negative, finite sequences of real numbers such that
\[
\sum_{i=1}^{p}a_i \geq \sum_{i=1}^{p}b_i.
\]
Then there exist a finite sequence of reals $c_i$ such that
\begin{itemize}
\item $\sum_{i=1}^p c_i=0.$
\item $a_i \geq b_i+c_i \  \forall i.$
\end{itemize}
\end{claim}


Now we proceed to do the proof. For  clusters $C_s\neq C_t$ define the matrix $H^{C_sC_t}$ whose entries are given by: 
\[
H^{C_iC_j}_{uv}= 
\begin{cases}
1 \text{ if } u,v \in C_s \text{ or } u,v \in C_t. \\
-1 \text{ if } u \in C_s, v \in C_t \text{ or } u \in C_s, v\in C_t. \\
0 \text{ in any other case.}
\end{cases}
\]
Notice that this matrix has $4$ blocs. Two with only $1$ and two with only $-1$. Moreover, its spectral norm is equal to $|C_s+|C_t|$.

\begin{lemma}\label{lemma: transport2}
Assume there are $l$ clusters. Suppose that $ b(\Gamma,\delta) < min(\delta,a(\Gamma,\delta))$.Then,
$A^*$ is a minimizer of the optimization problem $\min_A\Delta(A)$. 
\end{lemma}

\begin{proof}

First of all, observe that
\[
\frac{-2\delta}{n}H^* = \frac{-2\delta}{n}\frac{1}{l-1}\sum_{1\leq s<t\leq l}H^{C_sC_t}.
\]
For each, $C_s\neq C_t$ construct a transport matrix $\Delta_{st}$ as in the previous theorem, as to assure that $\Delta_{st}H^{C_sC_t} = H^{C_sC_t} \Delta_{st}=0$. This can be done since $H^{C_sC_t}$ takes into account only two clusters. Recall that the total amount of weight to be corrected is $b(\Gamma,\delta)$. Let $w_{s,t}$ the weight to be distributed from cluster $s$ to cluster $t$. In the notation of the previous theorem, $w_{st}$ is just the sum of the 
$\gamma_{l_p}^{ij}$ where $l_p\in C_s $ and $ij \in C_t $.


Let 

\[
C:= -\frac{2\delta}{n}H^* + \sum_{i<j}\Delta_{ij} = \frac{-2\delta}{n}\frac{1}{l-1}\sum_{1\leq i<j\leq l}H^{C_iC_j} + \sum_{i<j}\Delta_{ij}
\]

Finally, assume that for each $i<j$, $\frac{1}{l-1}||H^{C_iC_j}||\geq ||\Delta_{ij}||$.

Then,
\[
\begin{aligned}
\left\|C\right \| & = \left\| \frac{-2\delta}{n}\frac{1}{l-1}\sum_{1\leq i<j\leq l}H^{C_iC_j} + \sum_{i<j}\Delta_{ij}  \right \|  \\ & \leq \frac{2\delta}{n(l-1)}\sum_{1\leq i<j\leq l}\left \|H^{C_iC_j}+ \frac{n(l-1)}{2\delta} \Delta_{ij}\right \|  \\
& = \frac{2\delta}{n(l-1)}\sum_{1\leq i<j\leq l} \max (\left\|H^{C_iC_j}\right\|,\left\|\frac{n(l-1)}{2\delta}\Delta_{ij}\right\|)& \\
\end{aligned}
\]
Now notice that 
\[
\begin{aligned}
\delta \geq b(\Gamma,\delta) \text{ therefore } (l-1)n & \geq \frac{(l-1)n}{\delta} b(\Gamma,\delta) \\\text{ which implies that }  \sum_{i<j}\left \|H^{C_iC_j}\right \|  & \geq  \frac{(l-1)n}{\delta}\sum_{i<j}w_{i,j}   \\
 & \geq \frac{(l-1)n}{2\delta}\sum_{i<j}\left \|\Delta_{i,j} \right \|  .
\end{aligned}
\]
By the claim, we can assume without loss of generality that
for each $i<j$,
\[
\left \|H^{C_i,C_j}\right \| \geq  \frac{(l-1)n}{\delta}\left \|\Delta_{i,j} \right \|.
\]

This implies that the last sum reduces to
\[
\frac{2\delta}{n(l-1)}\sum_{1\leq i<j\leq l} \left\|H^{C_iC_j}\right\| = \sum_{1\leq i<j\leq l} \frac{2\delta(|C_i|+|C_j|)}{n(l-1)} = \frac{2\delta(l-1)}{n(l-1)}\sum_{1\leq i<j\leq l}|C_i|+|C_j|=2\delta.
\]
And so
$\|C\| \leq 2\delta$.

\end{proof}
\ddr{toca revisar que $<h^*,C>$ es igual a $-2\delta n$ o eso es obvio?}


\subsection*{Uniqueness of the minimizer}
In this brief section we discuss an important corollary: $A^*$ is the unique minimizer of the optimization problem $min_A\Delta (A)$.
We begin proving a well known lemma. 

\begin{lemma}{\label{lemUniqueness}}
Let f be a convex function defined over a region $D$. Let $\hat{x}$ be a point in it's domain such that the subdifferential of $f$ at $\hat{x}$ is full dimensional and $0$ belongs to it's interior. Then, $\hat{x}$ is the unique minimizer of $f$.
\end{lemma}
\begin{proof}
Let $B_\epsilon(0)$ be a ball of radius $\epsilon$ centered in $0$ and contained in 
$\partial(f)(\hat{x})$. Let $x \in D \text{ with } x \neq \hat{x}$. Let $Q \in \partial(f)(\hat{x})$. By the property of the elements of the subdifferential at a point, 
\[
f(x) \geq f(\hat{x}) + \left \langle Q,x-\hat{x} \right \rangle.
\]
This property holds for every $Q \in \partial(f)(\hat{x})$, so taking supremum we obtain that
\[
f(x) \geq \sup_{Q \in \partial(f)(\hat{x})} f(\hat{x}) + \left \langle Q,x-\hat{x} \right \rangle..
\]
Since $B_\epsilon(0) \subseteq \partial(f)(\hat{x})$ we obtain that

\[
 f(x) \geq f(\hat{x})+\sup_{Q \in B_\epsilon(0)} \left \langle Q,x-\hat{x} \right \rangle = f(\hat{x})+ \epsilon\| x-\hat{x} \|.
\]
As $x \neq \hat{x}$, $\epsilon \| x-\hat{x} \|> 0$. Therefore, $f(x)>f(\hat{x})$ for all $x\in D$ different of $\hat{x}$.

\end{proof}

\begin{remark}{\label{remUnicity}}
Under the conditions of theorem \ref{thm: transport}, The matrix $C^1$ we constructed satisfies the inequalities given by \ref{lem: subdiff} strictly, so $c^1$ is an interior point of $\partial(f)(A^*)$. It follows that the matrix $C$ constructed in \ref{lemma: transport2} also satisfies these inequalities strictly and thefore it is also an interior point of $\partial(f)(A^*)$.
\end{remark}




\begin{cor}
Under the conditions of theorem \ref{thm: transport}, $A^*$ is the unique minimizer of the optimization problem $min_A\Delta (A)$.
\end{cor}
\begin{proof}
By \ref{remUnicity}, the matrix $C$ constructed in \ref{lemma: transport2} is an interior point of  $\partial(f)(A^*)$. Therefore, as the subdifferential of $\Delta$ at $A^*$ is equal to the Minkowski sum of the subdifferentials of $g$ and $f$ at $A^*$,$0=C-C$ is an interior point of $\partial(\Delta)(A^*)$. It follows by lemma \ref{lemUniqueness} that $A^*$ is the unique minimizer of the optimization problem  $min_A\Delta (A)$.


\end{proof}



Using  Theorem \ref{thm: transport} we now estimate the probabilities of perfect recovery of the correct cluster structure. 



\subsection{A bound for recovery probabilities}

In this section we will bound the probability that the correct $A^*$ is not an optimal solution of our proposed optimization problem. A key tool will be the following version of Hoeffding's inequality: If $X_1,\dots, X_T$ are independent random variables with values in $[c_i,d_i]$ and $\Lambda_T:=\sum_{i=1}^T X_i$ then the following inequality holds for all $t\geq 0$ 
\[\PP\{\Lambda_T-\EE[\Lambda_T]\geq t\}\leq \exp\left(-\frac{2t^2}{\sum_{i=1}^T (d_i-c_i)^2}\right).\]




\begin{theorem} Suppose $B_1,\dots, B_N$ are independent and have the same distribution as $\grG$. If $\alpha = \min (|p_t-\frac{1}{2}|, |q-\frac{1}{2}|)$ and $\delta^*$ is the maximum of $a(\delta):=\frac{\delta\left(\alpha-\frac{\delta}{n}\right)^2}{\left(1+\frac{2\delta}{n}\right)}$ in $[0,\alpha n]$ then the probability that $A^*$ is not a minimizer of~(\ref{eqn: WRGS2}) is bounded above by
\[ \exp\left(-\frac{2N(n-1)^3(\alpha n-\delta^*)^2}{n}\right) + e^{-N\left(\delta^* a(\delta^*)\right)}\prod_{i\neq j}\left(1+ (e^{-4a}p_{ij}+q_{ij})^{\frac{N}{2}}\right).\] 
Moreover this quantity decreases exponentially with the sample size $N$.
\end{theorem}
\begin{proof} By Lemma~\ref{lemma: transport2} and the union bound the probability that $A^*$ is not an optimal solution of problem~(\ref{ProbRobusto}) is bounded above by

\begin{equation}\label{Eq: 2Probs}
\PP\{b(\Gamma,\delta)\geq a(\Gamma,\delta)\}+\PP\{b(\Gamma,\delta)\geq \delta\}.
\end{equation}

and we will find upper bounds for the individual terms in~(\ref{Eq: 2Probs}). 
For $t=1,\dots, N$ and $i,j\in [n]$ define 
\[Z^{(t)}_{ij}:=\begin{cases}
-1\text{, if $(B_t)_{ij}=1$}\\
+1\text{, if $(B_t)_{ij}=0$}
\end{cases}\]
and note that for every $i\neq j$ the equality $\sum_{t=1}^N\frac{Z_{ij}^{(t)}}{N} = \Gamma_{ij}$ holds. As a result
\[b(\Gamma,\delta)-a(\Gamma,\delta) = \sum_{i\neq j} \left(\widetilde{\Gamma_{ij}}+\frac{2\delta}{n}\right)= \frac{2\delta n(n-1)}{n} + \sum_{t=1}^N \sum_{i\neq j} \frac{\widetilde{Z^{(t)}_{ij}}}{N}\]
and therefore if $M:=\EE\left[\sum_{t=1}^N \sum_{i\neq j} \frac{\widetilde{Z^{(t)}_{ij}}}{N}\right]$ then the number $M$ is negative and is given by the formula
\[M=(2q-1)2|O|+\sum_{i=1}^l 2\binom{c_i}{2} (1-2p_i)\leq -2\alpha n(n-1)\]  
We can therefore bound the probability in the first term with
 
\[\PP\left\{\frac{2\delta n(n-1)}{n} + \sum_{t=1}^N \sum_{i\neq j} \frac{\widetilde{Z^{(t)}_{ij}}}{N}\geq 0\right\} = \PP\left\{\sum_{t=1}^N \sum_{i\neq j} \frac{\widetilde{Z^{(t)}_{ij}}}{N} - M \geq -2\delta (n-1) - M \right\}\leq\]
\[\leq \exp\left(-2\frac{(-M-2\delta(n-1))^2}{Nn(n-1)(\frac{2}{N})^2}\right)=\exp\left( -\frac{N}{2}\left(1-\frac{1}{n}\right) \left(\frac{-M}{n-1}-2\delta\right)^2\right) \]
where the inequality follows from Hoeffding's inequality applied to the $Nn(n-1)$ independent random variables $\frac{\widetilde{Z_{ij}^{(t)}}}{N}$ which have values in $\left[-\frac{1}{N},\frac{1}{N}\right]$. The inequality applies whenever $-\frac{M}{2(n-1)}>\delta>0$. In particular whenever $0<\delta<\alpha n$ we have
\[\PP\{b(\Gamma,\delta)-a(\Gamma,\delta)\}\leq \exp\left(-\frac{2N(n-1)^3(\alpha n-\delta)^2}{n}\right).\]
Bounding the second term is more involved. Recall that
\[b(\Gamma,\delta)=\sum_{i\neq j} \max\left(\widetilde{\Gamma_{ij}}+\frac{2\delta}{n},0\right).\]
Let $Y_{ij}:=\widetilde{\Gamma_{ij}}+\frac{2\delta}{n}$ and let $X_{ij}:=\max\left(Y_{ij},0\right)$. In order to prove a concentration inequality for the variables $X_{ij}$ we begin by studying their moment generating functions $m_{X_{ij}}(t)$. Note that for every real number $t$ the equality
\[\exp(tX_{ij})= 1_{\{Y_{ij}\leq 0\}} + 1_{\{Y_{ij}\geq 0\}} \exp tY_{ij}\]
holds. Now $Y_{ij}=1+\frac{2\delta}{n}-2\frac{n_{ij}}{N}$ where $n_{ij}$ is a binomial random variable with parameters $N$ and $p_{ij}$ given by
\[
p_{ij}:=\begin{cases}
p_t\text{, if $\{i,j\}\subseteq C_t$}\\
1-q\text{, else}
\end{cases}
\]


As a result taking expected values on both sides of the expression above we conclude that
\[ m_{X_{ij}}(t) \leq \PP\{Y_{ij}\leq 0\} + e^{t\left(1+\frac{2\delta}{n}\right)}\EE\left(e^{-\frac{2t}{N}n_{ij}}1_{\{Y_{ij}\geq 0\}}\right).\]

Using the Cauchy-Schwartz inequality and the known formula for the moment generating function of a binomial random variable it follows that
\[ m_{X_{ij}}(t)\leq \PP\{Y_{ij}\leq 0\}+ e^{t\left(1+\frac{2\delta}{n}\right)}\EE\left(e^{-\frac{4t}{N}n_{ij}}\right)^{\frac{1}{2}}\PP\{Y_{ij}\geq 0\}^{\frac{1}{2}}=\]
\[=\PP\{Y_{ij}\leq 0\}+ e^{t\left(1+\frac{2\delta}{n}\right)}\left(e^{-\frac{4t}{N}}p_{ij} + q_{ij}\right)^{\frac{N}{2}}\PP\{Y_{ij}\geq 0\}^{\frac{1}{2}}\]
where $q_{ij}:=1-p_{ij}$.
By Hoeffding's inequality on Bernoulli random variables we know that
\[\PP\{Y_{ij}\geq 0\}\leq \exp\left(-\frac{N}{2}\left(-\frac{2\delta}{n}-(1-2p_{ij})\right)^2 \right)\leq \exp\left(-2N(\alpha-\delta/n)^2\right)\]
so if $t=aN$ the inequality 
\[e^{t\left(1+\frac{2\delta}{n}\right)}\exp\left(-N(\alpha-\delta/n)^2\right)\leq 1\] 
holds whenever $a\leq \frac{(\alpha-\delta/n)^2}{\left(1+\frac{2\delta}{n}\right)}$ and for all such $a$ we have
\[m_{X_{ij}}(aN)\leq \left(1+ (e^{-4a}p_{ij}+q_{ij})^{\frac{N}{2}}\right)\]
We define $a(\delta):=\frac{(\alpha-\delta/n)^2}{\left(1+\frac{2\delta}{n}\right)}$ and will use it to prove a moment concentration inequality for $b(\Gamma,\delta)$ which will give us a bound on the second term in~(\ref{Eq: 2Probs}). For every $t>0$ we have

\[\PP\left\{b(\Gamma,\delta)\geq \delta\right\}=\PP\left\{\exp\left(t\sum_{i\neq j} X_{ij}\right)\geq e^{t\delta}\right\}\leq e^{-t\delta}\prod_{i\neq j} \EE[e^{tX_{ij}}]= e^{-t\delta}m_{X_{ij}}(t)\]
Choosing $t=a(\delta)N$ and using the previous inequality we see that
\[\PP\left\{b(\Gamma,\delta)\geq \delta\right\}\leq e^{-N\delta a(\delta)}\prod_{i\neq j}\left(1+ (e^{-4a}p_{ij}+q_{ij})^{\frac{N}{2}}\right)\]
Which decreases exponentially in $N$ for any $0\leq \delta\leq n\alpha$. The rate of decrease of the first term is controlled by the positive factor 
\[\delta a(\delta)= \frac{\delta\left(\alpha-\frac{\delta}{n}\right)^2}{\left(1+\frac{2\delta}{n}\right)}.\]
and we let $\delta^*$ be a maximizer of this function.
\end{proof}


\section{Uniqueness of the minimizer $A^*$}
\ddr{The following section is deprecated in favor of the alternative proof of uniqueness.}

In this section, we address the question of whether or not the matrix $A^*$ is the unique minimizer of the optimization problem $min_A\Delta(A)$. We begin by stating two lemmas.

\begin{lemma}
Let $C_1$ and $C_2$ be two convex sets in $R^m$ for some m. Denote the relative interior of a set $C$ by Relint(C) Then,
\[
Relint(C_1)+Relint(C_2) \subseteq Relint(C_1+C_2).
\]
\end{lemma}


\begin{lemma}{\label{lema2}} If $\langle N,Q\rangle\leq 0$ for every $Q$ in a convex set $C$ which contains 0 in its relative interior then the equality  $\langle N,Q\rangle=0$ holds for every $Q\in C$.
\end{lemma}
\begin{proof} Because of the inequality in our assumptions the set
\[F:=\{ Q\in C: \langle N,Q\rangle=0\}\]
is a face of the convex set $C$. 
Moreover this face contains $Q=0$ which is a point in the relative interior of $C$. Since the only face of a convex set which intersects its relative interior is $C$ itself we conclude that $F=C$ as claimed.
\end{proof}
We will now prove that with very high probability, $A^*$ is the unique minimizer of $min_A\Delta(A)$. this fact will follow from the next theorem. We begin with some notation.
Let $C$ denote the subdifferential of $\Delta$ at $A^*$. $\partial(\Delta(A))(A^*)$. since $\Delta(A)$ is a convex function, $C$ a convex region.
Recall that 
\[\Delta(A)= \delta\|2A-11^t\|_{*}+\frac{1}{N}\sum_{k=1}^N\|A-B_k\|_1\] 
And define functions $f$, $g$:
\[
f:= \frac{1}{N}\sum_{k=1}^N\|A-B_k\|_1, \ \ g:= \delta\|2A-11^t\|_{*}
\]
In this vocabulary, we have that
\begin{equation}\label{SumaSubDif}
C = \partial(f)(A^*)+\partial(g)(A^*)
\end{equation}

\begin{theorem}
Assume that $A^* \in argmin \Delta(A)$ and that $0 \in Relint(C)$. Then, $A^*$ is the unique minimizer of the optimization problem $min_A\Delta(A)$.
\end{theorem}
\begin{proof}
Suppose there exist a minimizer $\bar{A} \in argmin \ \Delta(A)$ with $\bar{A} \neq A^*$.
Let $N:=\bar{A}-A^* $. We will prove that $N=0$.
Let $Q \in C$. By definition of a subdifferential in $A^*$ we have that
\[
\Delta(\bar{A}) \geq \Delta(A^*) + \left \langle Q,  \bar{A}-A^* \rangle \right.
\]
Since both $\bar{A}$ and $A^*$ are minimizers of $\Delta(A)$ we have that $\Delta(\bar{A})=\Delta(A^*)$ and it follows that
\[
\left \langle Q,N \rangle \right. \leq 0.
\]
For all $Q \in C$. since $0 \in Relint(C)$ by hypothesis,  lemma \ref{lema2} implies that 
\[
\left \langle Q,N \rangle \right. = 0.
\]
For all $Q \in C$.

Now, observe the key following fact. By the characterizations of the subdifferentials $ \partial(f)(A^*) $ and $ \ \partial(g)(A^*)$ discussed in the previous section, we have that $\frac{-2\delta}{n}H^*  \in \partial(g)(A^*) $ and that $H^* \in \partial(f)(A^*)$. By (\ref{SumaSubDif}) we have that $H^*+\frac{-2\delta}{n}H^* \in C$ which implies that
\[
\left \langle H^* + \frac{-2\delta}{n}H^*,N \right\rangle= 0
\]
and 
\[
\left \langle H^*,N \right\rangle = \left \langle \frac{-2\delta}{n}H^*,N \right\rangle = 0.
\]

For $T \in \partial(f)(A^*)$ we have that $T+ \frac{-2\delta}{n}H^* \in C$ and
\[
0 = \left \langle T+ \frac{-2\delta}{n}H^*,N \right\rangle = \left \langle T,N \right\rangle + \left \langle \frac{-2\delta}{n}H^*,N \right\rangle =  \left \langle T,N \right\rangle.
\]

Using this same argument with $H^*$ instead of $\frac{-2\delta}{n}H^*$ we have that for all $S \in \partial(g)(A^*), \ \left \langle S,N \right\rangle =0.$

The two previous facts are the key to show that $N=0$. To prove this, we will carefully select some members of $\partial(f)(A^*)$ and of $\partial(g)(A^*)$ and use them to show that the entries of $N$ are all $0$.

First of all, let $D_i$ be a $n \times n$ diagonal matrix with  entries equal to 0 except in the entry $ii$ where we set it to $1>\epsilon>0$.

\[
D = 
 \begin{bmatrix}
    0 & & & \\
   & \epsilon  & \\
   & &\ddots & \\
   &  & & 0
  \end{bmatrix}
\]
Recall that the only restriction for a entry diagonal of a matrix in $\partial(f)(A^*)$ is that its absolute value is bounded by $n$ and that $H^* \in \partial(f)(A^*)$
The matrix $H^* + D_i$ has the same entries of $H^*$ except in the entry $ii$ where its value is $H^*_{ii}+ \epsilon = 1+\epsilon<n$ so that $H^* + D_i\in \partial(f)(A^*)$. 

Then,
\[
0 = \left \langle H^*+D_i,N \right\rangle = \left \langle H^*,N \right\rangle +  \left \langle D,N \right\rangle = \left \langle D,N \right\rangle = \epsilon N_{ii}.
\]
which implies that $N_{ii}=0$ for all $1\leq i\leq n$. Note that in particular $Tr(N)=0$. 

Let $G_I$ be the set of pairs $\{i,j\}$ such that $\{i,j\}\in I$, $i\neq j$ and $\frac{n_0(ij)-n_1(ij)}{N}< 1$.
Let $\{i,j\} \in G_I$ and chose $\epsilon>0$ small enough such that $\frac{n_0(ij)-n_1(ij)}{N}+\epsilon< 1.$
Now, define the matrix $T^{i,j}$ as a matrix of all zeros except at its entries $ij,ji$ where its value is set to $\epsilon$. I.e, $T^{i,j}$ is of the form

\[
T^{ij} = 
 \begin{bmatrix}
    0 & & & & \\
   & 0&  &\epsilon  & \\
   & &\ddots &  &\\
   &  \epsilon & & 0& \\
   &  &  &  & 0
  \end{bmatrix}
\]
Notice that the matrix $H^*+T^{ij}$ belongs to $\partial(f)(A^*)$ since all its entries all equal to those of $H^*$, except at $ij$ and $ji$, where its entries are $1-\epsilon$ and therefore

\[\frac{n_0(ij)-n_1(ij)}{N}\leq H_{ij}^*+T^{ij}_{ij} <1\]

We obtain that
\[
0 = \left \langle H^*+T^{ij},N \right \rangle = \left \langle H^*,N \right \rangle + \left \langle T^{ij},N \right \rangle = \left \langle T^{ij},N \right \rangle = \epsilon N_{ij} + \epsilon N_{ji}. 
\]
By the symmetry of $N$ it follows that $n_{ij}=n_{ji}=0 \ \forall  \ \{ij\} \in G_I$.
Define the set $G_O$ as the set of pairs $\{i,j\}$ such that $\{i,j\}\in O$, $i\neq j$ and $-1> \frac{n_0(ij)-n_1(ij)}{N}$.
Using the same argument as before with $-\epsilon$ instead of $\epsilon$, we can show that 
$n_{ij}=n_{ji}=0 \ \forall  \ \{ij\} \in G_O$.

It only remains to show that that entries of $N$ which are not in $G_I$ or in $G_O$ are $0$. Define the set $B_I$ as $I\setminus G_I$.
let $\{i,j\} \in B_I$. This means that $\{i,j\}  \in I$ but $\frac{n_0(ij)-n_1(ij)}{N}=1$.
Define the symmetric matrix $S^{ij}$ as follows:
\begin{equation}
S^{ij}_{kl} = 
\begin{cases}
0 \text{ if } \{k,l\}\in G_I \text{ or if } \{k,l\}\in G_O. \\
2\delta \text{ if } k=l \text{ but } k\neq i \text{ or } k\neq j. \\ 
2\delta \text{ if } \{k,l\}=\{i,j\}. \\
0 \text{ if } i=k=l \text{ or if } j=k=l.
\end{cases}
\end{equation}
In other words, $S^{ij}$ is of the form
\[
S^{ij} = 
 \begin{bmatrix}
    2\delta & & & & & & \\
   & 2\delta&  &  & & &\\
   & & 0&  &  &  2\delta& \\
   & & & 2\delta & & & \\
   & & & & \ddots & &\\
   &  & 2\delta & & & 0& \\
   &  & &  &  &  &2\delta
  \end{bmatrix}
\]
First of all, observe that $\{ij\} \in I $ so $H^*_{ij}=H^*_{ji}=1$, and as the entries in the diagonal of $H^*$ are also $1$, we have that
\[ \left \langle H^*,S^{ij} \right\rangle =2\delta(Tr(H^*)-H^*_{ii}-H^*_{jj}) + 2\delta H_{ij}+  2\delta H_{ji}  =(n-2) 2\delta+2(2\delta) =2\delta.\]

Moreover, the spectral norm of $S_{ij}$ is bounded by $\sqrt[]{\|S^{ij}\|_1 \|S^{ij}\|_{\infty}}$ by holder inequality. Here these norms are the induced $1$ norm and the induced $\infty$ norm. As the sum of  every row and every column of $S^{ij}$ is $2\delta$, we obtain that
\[\|S^{ij}\| \leq \sqrt[]{(2\delta)(2\delta)} = 2\delta.\]
We conclude that $S^{ij} \in \partial(g)(A^*)$ and so,

\[
0 = \left \langle S^{ij},N \right \rangle = 2\delta (Tr(N)-N_{ii}-N_{jj})+ 2\delta N_{ij} + 2\delta N_{ji} = 4\delta N_{ij}. 
\]
The last equality is given by the fact that $N$ is Symmetric. this means that 
$N_{ij}=N{ji}=0$ for all ${i,j} \in B_I$.
Finally, define $B_O$ as $O\setminus G_O$.
let $\{i,j\} \in B_O$. This means that $\{i,j\}  \in O$ but $\frac{n_0(ij)-n_1(ij)}{N}=-1$.
A similar argument using $-2\delta$ in the off diagonal entries shows that
$N_{ij}=N{ji}=0$ for all ${i,j} \in B_O$.

To summarize, we have shown that all entries of $N$ must equal $0$ since any entry belongs to its diagonal,$ G_I,G_O,B_I$ or $B_O$ are $0$ and these sets are a partition of the entries of $N$.
We conclude that $\bar{A}=A^*$.
\end{proof}



\end{document}


Our following Lemma gives us a decomposition Theorem for symmetric matrices. This decomposition will be very useful when computing spectral norms in the two-block case.
Suppose that we have only two blocks $C_1$ and $C_2$ and let $u$ be the vector given by
\[u_i=\begin{cases}
1\text{ if $i\in C_1$}
-1\text{ if $i\in C_2$}
\end{cases}
\]
Define $H^*:=uu^t$ and for $i=1,\dots, n$ let $e_{ii}$ be the $n\times n$ matrix whose only nonzero entry is in position $ii$ and has value $1$ 


\begin{lemma} Assume $|C_1|\neq |C_2|$ If $A$ is a symmetric matrix then there exist unique constants $\lambda_1,\dots, \lambda_{n-1}$ and a matrix $K$ such that $KH^*=H^*K=0$ such that
\[A= \frac{\langle H^*,A\rangle }{n^2} H^* +\sum_{i=1}^{n-1}\lambda_1(e_{i,i}-e_{i+1,i+1})+K.\]
And in particular the following equality holds  
\[ \|A\| =\max(A-K,K).\]
\end{lemma}
\begin{proof} The $n$ vectors in $v^tH^*$ and $v^t(e_{i,i}-e_{i+1,i+1})$ for $i=1,\dots, n-1$ are linearly independent in $\RR^n$ because the sum of the entries of all but the first one vanish. It follows that $v^tA$ can be expressed uniquely as a linear combination of them 


\end{proof}






\end{document}


